\createTaskHeader[Zúfalo málo miesta][3][7]
Matúš má problém s nedostatkom miesta na svojom počítači. Jeho uši sú ochotné počúvať len najkvalitnejšiu
hudbu v~bezstratovom formáte, oči mu krvácajú, ak uzrú video v kvalite horšej od Blu-ray a~so svojim
fotoaparátom spravil obrovské množstvo fotiek vo vysokom rozlíšení, ktoré odmieta zmazať. Zábavný priemysel
je neúprosný a~chudák Matúš už nemá svoje dáta kam uložiť.

Nezostáva mu teda nič iné, len sa opäť raz ponoriť do divokých vôd e-shopov a~rozšíriť pamäťovú kapacitu svojho počítača.
Stiahol si preto celú ponuku diskov z~najväčších internetových obchodov do jediného prehľadného súboru
a~teraz rozmýšľa, ktorý disk kúpiť. Matúš je ekonomicky cítiaci človek, a~tak by rád zistil, ktorý z diskov má najlepší pomer ceny ku kapacite.
Pomôžete mu s~tým? Najlepšie bude, ak pri tom použijete len celé čísla, pretože ostatným Matúš veľmi neverí.

\subsubsection{Úloha}

Na vstupe máte zoznam dostupných diskov v~obchodoch. Vašou úlohou je nájsť disk, ktorý je najvýhodnejší -- teda taký,
ktorý má najnižšiu jednotkovú cenu za gigabajt (teda spomínaný pomer ceny a kapacity).
Ak je najlepších diskov viacero, vypíšte ľubovoľný z~nich. Snažte sa vymyslieť taký algoritmus, ktorý pri výpočtoch
používa len celé čísla. Najlepšie bude, ak nebudete vôbec nikde deliť.

\subsubsection{Formát vstupu}

V~prvom riadku vstupu je kladné číslo $n$ udávajúce počet diskov.
Vkaždom znasledujúcich $n$ riadkov sú celé čísla $c$ a~$k$ udávajúce cenu (v~eurách)
a~kapacitu (v~gigabajtoch) daného disku.

\subsubsection{Formát výstupu}

Vypíšte dve medzerami oddelené čísla -- cenu a~kapacitu hociktorého najvýhodnejšieho disku v~zozname.
Nezabudnite za nimi vypísať koniec riadku.

\subsubsection{Hodnotenie}

Za popis riešenia, ktoré používa aj iné ako celé čísla sa dajú získať najviac 3 body zo 7.
Počet bodov za program závisí len od toho, ktoré vstupy váš program vyrieši správne.
Vstupy sú rozdelené do sád podľa obtiažnosti, za každú sadu sa dá získať pol boda, ale celkový počet bodov za program sa zaokrúhľuje nadol.

\begin{table}[ht]
    \begin{center}
        \begin{tabular*}{0.8\textwidth}{@{\extracolsep{\fill}} l *{6}{r} @{\extracolsep{\fill}}}
            \toprule
                \textbf{Sada} & 1 & 2 & 3 & 4 & 5 & 6 \\
            \midrule
                \textbf{Maximálny počet}     &   2 &   3 &    10 & 1 000 & 10 000 & 100 000 \\
                \textbf{Maximálna cena}      & 100 & 100 & 1 000 & 1 000 & 10 000 &  10 000 \\
                \textbf{Maximálna kapacita}  & 100 & 100 & 1 000 & 1 000 & 10 000 &  10 000 \\
            \bottomrule
        \end{tabular*}
    \end{center}
\end{table}

\subsubsection{Príklad}

\exampleIO{
5
500 100
750 130
1500 200
250 60
1000 147
}{
4
}
