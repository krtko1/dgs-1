\createTaskHeader[Zwarte doos][40]
Keď sa Syseľ s Majkou vracali domov z Ameriky, museli absolvovať
niekoľkohodinový prestup v Amsterdame. No a aby len tak bezcieľne
nesedeli na letisku, rozhodli sa tento čas využiť na prechádzku po
okolí.

Okrem mnohých obchodov špecifických pre tento kraj sa zastavili aj v
obchode so starožitnosťami. Kým sa Majka nadchýnala keramikou, Syseľ
našiel niečo zaujímavejšie -- malú čiernu skrinku s ciferníkom. Nastavil
na nej číslo 27 a natiahol ju kľúčikom uprostred. Skrinka chvíľku
hrkotala a potom sa čísla na ciferníku prestavili na 55. Syseľ skúsil 42
a dostal 85. Celý nadšený z toho, čo našiel, išiel za Majkou. Predavač
si všimol jeho nadšenie pre skrinku a dal mu zaujímavú ponuku. Ak príde
na to, ako skrinka funguje a donúti ju zobraziť číslo 47, tak si ju môže
nechať. Vraj o ňu aj tak nik iný nemá záujem.

Syseľ s Majkou samozrejme zistili ako skrinka funguje a odniesli si ju
domov. Dokonca našli na skrinke prepínač, ktorý mení jej funkciu. Majú
teda kopu zaujímavých hlavolamov, o ktoré sa s vami chcú podeliť.

\subsubsection{Úloha}

Dostanete prístup k simulátoru čiernej skrinky. Vždy, keď do nej vložíte
nejaké číslo, skrinka niečo vypíše (číslo, slovo\dots) podľa
jednoduchého pravidla.

Vašou úlohou bude toto pravidlo odhaliť a nájsť vstupné číslo také, aby
skrinka vypísala požadovanú vec.

Hra má 10 úrovní, za každú získate bod. Náročnosť stúpa spolu s
úrovňami, môžete ich však riešiť v ľubovoľnom poradí.

\subsubsection{Odovzdávanie a bodovanie}

Za každú vyriešenú úroveň dostanete jeden bod. K tejto úlohe netreba
odovzdávať žiadny popis ani program. Simulátor spolu so všetkými
potrebnými informáciami o odovzdávaní nájdete na stránke
\url{http://ksp.sk/hry/32/1/1}
