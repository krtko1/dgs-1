\createTaskHeader[Poistka vs. Li-Ion články]
Jaro sa hral s~$\SI{300}{\ampere}$ poistkou a~čuduj sa svete, podarilo sa mu ju prepáliť :).
K~dispozícii mal ľubovoľne veľa Li-Ion článkov s~elektromotorickým napätím $\SI{3.3}{\volt}$ a~vnútorným odporom $\SI{10}{\milli\ohm}$.
Vieme ich spájať pomocou ľubovoľného počtu dokonalých vodičov s~nulovým odporom do ľubovoľne veľkého zdroja.
Na jeho konce pripojíme poistku s~odporom $\SI{10}{\milli\ohm}$. Nájdite všetky také zapojenia,
pozostávajúce z~čo najmenšieho počtu článkov, ktoré prepália poistku.