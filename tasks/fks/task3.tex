\createTaskHeader[Surfovanie v~MHD]
Spomínate si na príklad z~minulého semestra s~cestujúcim Vladkom v~39-tke? Tak tento príklad je jeho pokračovaním.
Tentoraz Maťo, Mišo a~Kubo cestovali v~pražskom metre. Ako iste tušíte, opäť raz tam nezostalo veľa miesta, takže sa nemali čoho chytiť.
Ani koeficient trenia medzi stropom a~rukou nebol veľmi priaznivý a~oni museli vymyslieť nový spôsob, ako udržať rovnováhu.
Od vás by radi vedeli, či spadnú alebo nie.

Maťo sa postaví bokom k~smeru jazdy. Na podlahu vždy pôsobí iba svojou tiažou.
Presúvaním ťažiska v~konštantnej výške medzi rozkročenými nohami z~jednej nohy na druhú sa snaží udržať rovnováhu.
Maťo váži \SI{81}{\kilo\gram}, je vysoký \SI{185}{\centi\metre} a~môžete predpokladať, že jeho ťažisko je v~$3/5$ jeho výšky.
Vzdialenosť medzi jeho nohami je \SI{40}{\centi\metre} a~koeficient trenia medzi topánkami a podlahou metra je \num{0.5}.
Maťa by teraz zaujímalo, pre aké hodnoty zrýchlenia sa mu podarí udržať rovnováhu presúvaním ťažiska z~jednej nohy na druhú.

\Picture{task3}{4cm}[Maťovo ťažisko]
