Ahoj!

Sme študenti Fakulty matematiky, fyziky a informatiky
Univerzity~Komenského v~Bratislave. Prinášame Ti výnimočnú súťaž venovanú
žiakom základných škôl, ktorých zaujíma svet okolo nás, takže veríme,
že si jedným z~nich.

Úlohy, ktoré práve držíš v~rukách, od Teba nevyžadujú znalosti vzorcov
alebo poučiek, ale tvorivý prístup a chuť zamyslieť sa nad zaujímavým
problémom. Často bude úlohou zistiť, ako fungujú veci a zariadenia okolo
nás, vyrobiť a vyskúšať fyzikálny experiment alebo podumať, prečo sa veci
okolo nás dejú tak, ako sa dejú.

Takže ak aj nevynikáš znalosťami z~fyziky, ale zaujíma Ťa svet okolo
Teba a nebojíš sa roztočiť svoje mozgové závity, nečakaj s~riešením už
ani sekundu\dots A~ako vlastne súťažiť?

Celé to prebieha korešpondenčnou formou. Riešenia týchto úloh (to znamená
\emph{celý postup} riešenia a vysvetlenie, nie len výsledok) nám pošli
poštou do stanoveného termínu na adresu v~hlavičke. My Tvoje riešenia
opravíme, obodujeme a spolu so vzorovými riešeniami a novými úlohami Ti
pošleme späť. Takto prebehnú do januára tri série súťaže, na základe
ktorých súťaž vyhodnotíme.

Tých úplne najlepších odmeníme hodnotnými cenami a všetkých úspešných riešiteľov
pozveme na sústredenie. Je to týždňová akcia, ktorá sa uskutoční v~niektorej zo
slovenských škôl v~prírode. Popri prednáškach a seminároch venovaných fyzike na
nej zažiješ skvelú zábavu, akčné hry, večery pri gitare, nechýbajú ani divadlá,
noví kamaráti a zaujímavé zážitky. Hlavne však spoznáš skvelých ľudí! Ak aj
fyzika nebola vždy Tvojou obľúbenou disciplínou, zistíš, že fyzici sú super.

Všetky informácie o~UFO, debatu a~fotky zo sústrediek nájdeš na \URL{\seminarURL}, resp. \URL{http://www.fks.sk/}.

\hfill \textit{ Veľa zdaru Ti prajú Tvoji vedúci!}%
