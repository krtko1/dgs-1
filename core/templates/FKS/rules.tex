\pagestyle{rules}

\section{Pravidlá a postihy}
    \begin{itemize}
        \item Riešiteľom môže byť každý študent strednej školy, gymnázia, prípadne aj základnej školy.
        
        \item Od zverejnenia zadaní série na našej stránke Ti garantujeme aspoň 3 týždne času na riešenie.
    
        \item Za každý príklad môžeš získať najviac 9 bodov. 
    
        \item Súťaž prebieha v~dvoch kategóriach s~označením~\textbf{A} (pre ostrieľanejších) a~\textbf{B} (začínajúcich strelcov).
    
        \item To, ktorá kategória a príklady sú pre teba určené, vieš zistiť na základe svojho \textbf{FKS koeficientu}. 
            Ten nájdeš nasledovným spôsobom:
            $$3 - \text{počet rokov do maturity} + \text{počet úspešných semestrov.}$$
            Semester sa považuje za úspešný, ak si sa v~ňom umiestnil v~prvej trojke riešiteľov v~aspoň jednej z~kategórií A,~B 
            alebo ak si sa počas neho zúčastnil sústredenia FKS.
        
        \item \textbf{V~kategórii B} môžeš súťažiť, ak tvoj koeficient \textbf{nie je väčší ako 3} 
            a~ešte si sa nezúčasnil celoštátneho kola Fyzikálnej olympiády, pričom:
            \begin{itemize}
                \item ak je tvoj koeficient väčší než~0, môžeš riešiť príklady s~číslami 2~až~5;
                \item ak je tvoj koeficient menší alebo rovný~0, môžeš riešiť aj príklad číslo 1 a~do hodnotenia sa ti započítajú \emph{štyri najlepšie vyriešené príklady}.
            \end{itemize}
    
        \item \textbf{V~kategórii A} môžu súťažiť všetci riešitelia. Hodnotia sa v~nej príklady s~číslom 4~až~7.
    
        \item Do pozornosti dávame aj špeciálnu kategóriu FX (\URL{http://fks.sk/fx}),
            ktorá je určená skutočným labužníkom. Výsledky z~FX sa k~celkovým FKS výsledkom 
            pripočítavajú iba vtedy, ak si riešil kategóriu \textbf{A}~a~dosiahol si aspoň \emph{25\% bodov} najlepšieho riešiteľa.
            Tým sa po skončení korešpondenčnej časti k~normálnym bodom \emph{pripočítajú body
            získané v~FX} za príslušné obdobie. Okrem toho máme pre teba ďalšiu skvelú ponuku.
            Ak samostatne v~kategórii FX dosiahneš v~danom polroku aspoň \emph{polovicu plného počtu
            bodov}, budeš \emph{automaticky pozvaný} na sústredenie, aj keby si v~žiadnej FKS
            výsledkovke nevystupoval.
    
        \item[$\skull$] Úlohy rieš samostatne! Za odpisovanie strhávame body a~sme agresívni. 
    
        \item[$\skull$] Príklady posielaj načas! Dôležitý je \emph{termín odoslania} riešení (ak posielaš poštou, tak rozhoduje pečiatka odoslania).
            Za jeden deň po termíne ti strhneme 25\% tvojich získaných bodov zaokrúhlených nadol. 
            Riešenia odovzdané neskôr už neakceptujeme. Ak kvôli nejakému závažnému dôvodu nestíhaš odoslať riešenia načas, neváhaj nás kontaktovať na \URL{\seminarEmail}
            a vyriešime to individuálne.
    \end{itemize}

\subsection{Ako má vyzerať moje riešenie?}
    \begin{itemize}
        \item Ak preferuješ posielanie poštou, píš každý príklad na \emph{osobitný papier} A4. Viacstranové riešenie nezabudni
            zopnúť spinkou. Na každý papier napíš hore \emph{hlavičku} s~\textbf{menom}, \textbf{triedou}, \textbf{školou} a~\textbf{číslom príkladu}.
            Inak u~nás vo FKS zavládne chaos!
        \item Je iba málo vecí, ktoré vedia priviesť opravovateľa, ktorý je v~časovom strese, do stavu nepríčetnej zúrivosti. Nečitateľné riešenia to však dokážu perfektne.
            Pokiaľ nevieš písať čitateľne a~táto choroba sa u~vás dedí po generácie, skús pouvažovať o~písaní na počítači.
        \item Pri elektronických riešeniach je ideálny formát \texttt{pdf}, prežijeme aj \texttt{doc} a~\texttt{docx}.
            Ak však používaš OpenOffice, ukladaj súbory do \texttt{doc}, nie do \texttt{odt}.
            Jednotlivé príklady odovzdaj prostredníctvom našej stránky.
    \end{itemize}

\subsection{Ako získať za moje riešenie veľa bodov?}
    Ako v~mnohých iných súťažiach, aj tu platí jednoduchá zásada -- písať všetko, čo
    o~príklade vieš. Teda, aj keď nevieš celé riešenie, oplatí sa spísať aspoň časti
    riešenia (názory, postrehy, pokusy, náčrty). Pokiaľ však o~svojom riešení vieš, že
    nie je úplné, určite to napíš!

    Neprepadaj panike! Ak príklad nevieš vyriešiť, pravdepodobne to znamená, že je
    ťažký. Ak je ťažký pre teba, tak je zrejme ťažký aj pre iných. Nikto nevraví,
    že musíš byť v~prvej trojke. Aj 12. miesto je úspech -- minimálne z~hľadiska
    pozvania na sústredko.

    Opravovateľ je (väčšinou) tiež len človek a občas sa stane, že mu geniálna
    myšlienka v~tvojom riešení unikne. Ak máš pocit, že si obeťou konšpirácie a
    nezmyselnej lobby zameranej na poškodenie tvojej osoby, napíš k~príkladu pár
    milých slov (podľa možností niečo pádnejšie ako „Chcem viac bodov!“) a~pošli
    ho späť. Jednoduchšie však bude, ak napíšeš mail.

    Neopisuj! Po prvé, je to nemorálne. Po druhé, aj tak na to prídeme. Ak sa dve riešenia
    líšia iba farbou pera, nedementný opravovateľ si to nabetón všimne. Ak aj vám
    priamo nebudú strhnuté body, budeme vás ohovárať na priedomku.

    Ak nepochopíš úplne presne zadanie príkladu, môžeš sa nás na podrobnosti opýtať
    e-mailom na {\URL{\seminarEmail}}. Ak by bola v~príklade nejaká vážnejšia
    nejasnosť, nedajbože chyba v~zadaní, určite Ťa na to mailom upozorníme a taktiež to uverejníme na stránke.

\subsection{Chcem začať riešiť! Čo mám spraviť?}
    Zaregistruj sa na \URL{\seminarURL} a~začni čo najskôr počítať. Nečakaj však, že všetko stihneš porátať
    za posledný víkend či hodinu slovenčiny v~deň termínu odoslania série :-)

    \textbf{Zaregistruj sa, aj keď si už FKS riešil, alebo budeš posielať riešenia poštou.}
    Ak máš účet na seminároch Prask alebo KSP, ďalší si nezakladaj, účty sú totožné pre všetky semináre.

\hfill \emph{Veľa zdaru Ti prajú Tvoji vedúci!}%
