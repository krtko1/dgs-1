% Document Generation System - skrátene DGS, čítaj DeGeŠ, chce nahradiť všetky existujúce
% spôsoby sádzania trojstenových zadaní, vzorákov a ostatných dokumentov.

% Prosíme o dodržiavanie nasledovných zásad.

% NEVYMÝŠĽAJ VLASTNÉ HACKY, ŠPECIÁLNE NIE V PRÍPADE, ŽE TVOJ KÓD BUDE POUŽÍVAŤ AJ NIEKTO INÝ.
% Aj keď je TeX miestami nechutný, zatiaľ sa všetko podarilo spraviť +- čisto.

\RequirePackage[
    paper                   = a4paper,
    left                    = 15mm,
    right                   = 15mm,
    top                     = 15mm,
    bottom                  = 15mm,
    headheight              = 16pt,
    headsep                 = 16pt,
    footskip                = 32pt,
    includeheadfoot,                                        % we wish to include header and footer into page dimensions
    %showframe                                              % display visual frame (must be turned off for production)
]{geometry}


\RequirePackage{
    amsmath,
    siunitx,
    xparse,
    enumitem,
    graphicx,
    listings,
    lastpage,
    marvosym,
    calc,
    pdftexcmds,
    xifthen,
    verbatim,
    color,
    float,
    caption,
    booktabs,
    titlesec,
    afterpage,
    pifont,
    pgffor,
    longtable,
    csvsimple,
    animate,
    skull,
    etoolbox,
    MnSymbol,
    fancyhdr,
    hyphenat,
    fancyvrb                                                % fancy verbatim (currently unused)
}

\definecolor{colour-url}{RGB}{0, 137, 162}
\definecolor{colour-link}{RGB}{0, 137, 162}
\sisetup{
    detect-all              = true,                         % chceme zachovávať formátovanie prostredia
    separate-uncertainty    = true,                         % 7.2 ± 0.5
    multi-part-units        = single,                       % 12.3 ± 0.2 kg, neopakovať "kg"
    per-mode                = reciprocal,                   % `symbol` pre "m/s", `reciprocal` pre "ms^{-1}"
    group-separator         = {\,},
    group-minimum-digits    = 5,
    inter-unit-product      = {\kern 0.05em},               % veľmi tenká medzera medzi jednotkami
    exponent-product        = \times,                       % \times pre 5×10^7, \cdot pre 5.1O^7    
    number-unit-product     = {\ },                         % medzera medzi číslom a jednotkou
    output-decimal-marker   = {\text{,}},                   % v desatinných číslach chceme slovenskú čiarku
    range-units             = single,
    range-phrase            = {\text{ -- }},                  % 5 -- 10 m/s
    list-units              = single,                       %
    list-final-separator    = {\ a\ },                      % 5, 7 a 10 m/s
    retain-explicit-plus    = true,                         % zachovávame explicitné +
}

%
\RequirePackage[
    colorlinks              = true,
    linkcolor               = colour-link,                  % custom Trojsten link colour
    urlcolor                = colour-url,                   % custom Trojsten URL link colour
]{hyperref}

\RequirePackage[f]{esvect}                                  % nicer vector overset
\RequirePackage[all]{nowidow}

% Setup XeLaTeX-specific options

\RequirePackage{boolexpr}                                   % boolean expressions, switch

% Setup fonts -- see fontspec/mathspec documentation.
% Fonts are loaded from .ttf and .otf files in the core/fonts/ directory. We NEVER use system fonts.

% Change allocator to allow more than 16 alphabets per document

\makeatletter
    \def\new@mathgroup{\alloc@8\mathgroup\mathchardef\@cclvi}
    \patchcmd{\document@select@group}{\sixt@@n}{\@cclvi}{}{}
    \patchcmd{\select@group}{\sixt@@n}{\@cclvi}{}{}
\makeatother

\defaultfontfeatures{
    Mapping         = tex-text,
    Scale           = MatchLowercase,
    Ligatures       = TeX
}
\setallmainfonts[
    Path            = core/fonts/ ,
    Extension       = .otf ,
    UprightFont     = *-r ,
    BoldFont        = *-b ,
    ItalicFont      = *-i ,
    BoldItalicFont  = *-bi
]{minionpro}
\setmathfont(Digits,Latin,Greek)[
    Path            = core/fonts/ ,
    Extension       = .otf ,
    UprightFont     = *-r ,
    BoldFont        = *-b ,
    ItalicFont      = *-i ,
    BoldItalicFont  = *-bi
]{minionpro}
\setmonofont[
    Path            = core/fonts/ ,
    Extension       = .ttf ,
    UprightFont     = *-r ,
    BoldFont        = *-b ,
    ItalicFont      = *-i ,
    BoldItalicFont  = *-bi
]{ubuntumono}
\setsansfont[
    Path            = core/fonts/ ,
    Extension       = .ttf ,
    UprightFont     = *-r ,
    BoldFont        = *-b ,
    ItalicFont      = *-i ,
    BoldItalicFont  = *-bi
]{verdana}

% Fixes mathspec bug -- URL numbers are normally rendered with upright font
\makeatletter
    \DeclareMathSymbol{0}{\mathalpha}{\eu@DigitsArabic@symfont}{`0}
    \DeclareMathSymbol{1}{\mathalpha}{\eu@DigitsArabic@symfont}{`1}
    \DeclareMathSymbol{2}{\mathalpha}{\eu@DigitsArabic@symfont}{`2}
    \DeclareMathSymbol{3}{\mathalpha}{\eu@DigitsArabic@symfont}{`3}
    \DeclareMathSymbol{4}{\mathalpha}{\eu@DigitsArabic@symfont}{`4}
    \DeclareMathSymbol{5}{\mathalpha}{\eu@DigitsArabic@symfont}{`5}
    \DeclareMathSymbol{6}{\mathalpha}{\eu@DigitsArabic@symfont}{`6}
    \DeclareMathSymbol{7}{\mathalpha}{\eu@DigitsArabic@symfont}{`7}
    \DeclareMathSymbol{8}{\mathalpha}{\eu@DigitsArabic@symfont}{`8}
    \DeclareMathSymbol{9}{\mathalpha}{\eu@DigitsArabic@symfont}{`9}
\makeatother

\DeclareFontFamily{U}{skulls}{}
\DeclareFontShape{U}{skulls}{m}{n}{ <-> skull }{}
\newcommand{\skull}{\text{\usefont{U}{skulls}{m}{n}\symbol{'101}}}

\makeatletter

% Fixes mathspec bug -- URL numbers are rendered with wrong font
    \DeclareMathSymbol{0}{\mathalpha}{\eu@DigitsArabic@symfont}{`0}
    \DeclareMathSymbol{1}{\mathalpha}{\eu@DigitsArabic@symfont}{`1}
    \DeclareMathSymbol{2}{\mathalpha}{\eu@DigitsArabic@symfont}{`2}
    \DeclareMathSymbol{3}{\mathalpha}{\eu@DigitsArabic@symfont}{`3}
    \DeclareMathSymbol{4}{\mathalpha}{\eu@DigitsArabic@symfont}{`4}
    \DeclareMathSymbol{5}{\mathalpha}{\eu@DigitsArabic@symfont}{`5}
    \DeclareMathSymbol{6}{\mathalpha}{\eu@DigitsArabic@symfont}{`6}
    \DeclareMathSymbol{7}{\mathalpha}{\eu@DigitsArabic@symfont}{`7}
    \DeclareMathSymbol{8}{\mathalpha}{\eu@DigitsArabic@symfont}{`8}
    \DeclareMathSymbol{9}{\mathalpha}{\eu@DigitsArabic@symfont}{`9}

    \NewDocumentCommand{\padzero}{m m}{\@anim@pad{#1}{#2}}

% Disgusting hacks to bypass idiotic limitations of LaTeX -- 16 alphabets per document
    \def\new@mathgroup{\alloc@8\mathgroup\mathchardef\@cclvi}
    \patchcmd{\document@select@group}{\sixt@@n}{\@cclvi}{}{}
    \patchcmd{\select@group}{\sixt@@n}{\@cclvi}{}{}

% Render percent sign with nice font, not ugly Computer modern
    \mathcode`\%="7025

\makeatother

% Fix Pandoc's tightlist problem
\providecommand{\tightlist}{\setlength{\itemsep}{0pt}\setlength{\parskip}{0pt}}

\NewDocumentCommand{\URL}{m}{\href{#1}{\texttt{#1}}}

% package siunitx -- custom units
\DeclareSIUnit{\gforce}{g}
\DeclareSIUnit{\year}{y}
\DeclareSIUnit{\au}{AU}
\DeclareSIUnit{\pixel}{px}
\DeclareSIUnit{\year}{y}
\DeclareSIUnit{\eur}{€}

\NewDocumentCommand{\errorMessage}{+m}{\colorbox{red}{#1}}
\NewDocumentCommand{\todoMessage}{+m}{\colorbox{cyan}{#1}}

% Protected input -- \input entire file or typeset a warning message saying that it does not exist
\NewDocumentCommand{\protectedInput}{m}{%
    \IfFileExists{#1}{%
        \input{#1}%
    }{%
        \errorMessage{MISSING FILE \texttt{#1}!}\\%
    }%
}

\NewDocumentCommand{\cutHere}{}{%
    \noindent%
    \raisebox{-2.8pt}[0pt][0.75\baselineskip]{\small\ding{34}}%
    \unskip{\tiny\dotfill}
}


% \insertPicture{filename}{extension}{(ignored)}{height}{caption}{label}
% Insert a picture, protected against nonexistent files
%   - meno súboru, hľadá v aktuálnom priečinku
%   - prípona súboru
%   - prípona súboru pre HTML (ignoruje sa)
%   - výška obrázka (napríklad v mm)
%   - popis pod obrázkom
%   - label pre odkázanie sa na obrázok
\NewDocumentCommand{\insertPicture}{m m m m m m}{%
    \begin{figure}[H]%
        \centering%
        \IfFileExists{\problemDirectory/#1.#2}{%
            \includegraphics[keepaspectratio = true, height = #4]{\problemDirectory/#1.#2}%
        }{%
            \includegraphics[keepaspectratio = true, height = #4]{example-image-a}%
        }%
        \ifstrempty{#5}{}{\caption{\textit{#5}}}%
    \end{figure}%
}

\NewDocumentCommand{\insertPictureSimple}{m m m m}{%
    \begin{figure}[H]%
        \centering%
        \IfFileExists{\problemDirectory/#1}{%
            \includegraphics[keepaspectratio = true, height = #2]{\problemDirectory/#1}%
        }{%
            \includegraphics[keepaspectratio = true, height = #2]{example-image-a}%
        }%
        \ifstrempty{#3}{}{\caption{\textit{#3}}}%
    \end{figure}%
}

% \exampleIO[verbatim input][verbatim output]
% Príklad vstupu/výstupu (KSP)
%   - verbatim vstup
%   ₋ verbatim výstup
% Používa ExplSyntax z LaTeX3
\ExplSyntaxOn
\char_set_catcode_other:n {`\^^M}
\NewDocumentCommand{\exampleIO}{+v +v}{
    \tl_set:Nn \l_tmpa_tl {#1}
    \tl_set:Nn \l_tmpb_tl {#2}
    \tl_replace_all:Nnn \l_tmpa_tl {^^M} {\par}
    \tl_replace_all:Nnn \l_tmpb_tl {^^M} {\par}
    
    \begin{minipage}[t]{0.48\linewidth}     
        \begin{center}vstup\end{center}
        \vspace{-15pt}                                              % one nasty hack here       
        \fbox{
            \begin{minipage}[t]{\linewidth}
                \mbox{}\\[-1.5\baselineskip]                        % another nasty hack here
                \ttfamily
                \l_tmpa_tl
            \end{minipage}
        }
    \end{minipage}
    
    \begin{minipage}[t]{0.0393\linewidth}
        \mbox{}
    \end{minipage}
    
    \begin{minipage}[t]{0.48\linewidth}
        \begin{center}výstup\end{center}
        \vspace{-15pt}                                              % guess what here
        \fbox{
            \begin{minipage}[t]{0.975\linewidth}
                \mbox{}\\[-1.5\baselineskip]                        % and one nasty hack here
                \ttfamily
                \l_tmpb_tl
            \end{minipage}
        }
    \end{minipage}
    \\[1ex]
}
\char_set_catcode_end_line:n{`\^^M}
\ExplSyntaxOff

\NewDocumentCommand{\rootDirectory}{}{input}
\NewDocumentCommand{\problemDirectory}{}{<undefined>}

\endinput                                                           % everything below is safely ignored
