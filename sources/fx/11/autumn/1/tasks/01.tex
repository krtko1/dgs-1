\createTaskHeader[Kotúľanie loptičiek]
Miro sa začal hrávať s~loptičkami a~pri tom si všimol, že kotúľanie loptičiek bez prešmykovania
po vodorovnej podložke vlastne vôbec nemusí byť také nezaujímavé, ako sa môže zdať.
\begin{enumerate}[label=\alph*)]
	\item Štandardný spôsob, ako sa môže kotúľať loptička bez prešmykovania, je taký,
		že množina dotykových bodov s~podložkou tvorí kružnicu s~polomerom samotnej loptičky.
		Vašou úlohu v~tejto podúlohe je zistiť a~správne odargumentovať,
		či sa môže loptička kotúľať bez prešmykovania po podložke aj inými spôsobmi.
	\item Uvažujme loptičku kotúľajúcu sa možno nejakým neštandardným spôsobom ako z~predchádzajúcej podúlohy.
		Je možné, aby sa zmenil smer vektora rýchlosti ťažiska gule počas pohybu?
		Svoju odpoveď zdôvodnite aj formálnym výpočtom!
	\item Uvažujme znovu loptičku, ktorá sa pohybuje bez prešmykovania po podložke,
		znovu možno nejakým neštandardnejším spôsobom. Na podložke sa nachádza aj hárok papiera.
		Keď sa loptička dostane na papier, potiahneme ho v~ľubovoľnom (vodorovnom) smere tak rýchlo,
		že môže dôjsť k~prešmykovaniu loptičky vzhľadom na hárok papiera. Ukážte, že po tom, ako loptička
		zíde z~povrchu papiera a~ustáli sa jej rýchlosť, je rýchlosť loptičky rovnaká ako predtým,
		než sa loptička dostala na povrch hárku papiera!
\end{enumerate}
