\createTaskHeader[Molekuly na povrchu kryštálu]
Bzdušo sa hral s~molekulami absorbovanými na povrchu pevných kryštálov pri nízkych teplotách.
Spozoroval, že molekuly sa typicky zorganizujú do pravidelnej štruktúry, napr. takej,
že jedna molekula leží v~strede štvorca s~dĺžkou hrany $a$.
Takéto štvorce pripomínajú na povrchu kryštálu šachovnicu.
Bzdušo si všimol, že dvojatómové molekuly sú absorbované takým spôsobom,
že ich dlhšia os je rovnobežná s~povrchom a~orientácia každej molekuly je
určená elektrostatickou interakciou so susednými molekulami.

\begin{enumerate}[label=\alph*)]
	\item Molekula \textbf{CO} má malý elektrický \textbf{dipólový moment} $\vec{p}$.
		Na obrázku sa nachádza časť šachovnice, kde je orientácia dipólových momentov parametrizovaná uhlom $\alpha$.
		Uvažujme interakciu každej molekuly -- dipólu, len s~jej najbližšími ôsmimi susedmi.
		Pomôžte Bzdušovi nájsť uhol $\alpha$, ktorý minimalizuje celkovú energiu $\mathrm{E}_\text{min}$
		a~nájdite veľkosť tejto energie.
		
		\insertPicture{02-sachovnica.png}{20mm}[Usporiadanie do šachovnice na povrchu kryštálu][fig:1]

	\item Molekuly $\text{N}_2$ majú malý \textbf{kvadrupólový moment}, pretože kovalentné väzby vytvárajú
		malý negatívny náboj v~oblasti medzi atómovými jadrami. Na základe tohto faktu a~kvalitatívneho odhadu
		správanie skúste nakresliť, ako vyzerá orientácia molekúl $\text{N}_2$ na šachovnici.
\end{enumerate}
