\createTaskHeader[Asteroid a jadro]
\emph{Táto úloha je o~deformovaní telies zložených z~izotropného materiálu (t.j. takého, ktorého vlastnosti sú
vo všetkých smeroch rovnaké). Keďže potrebné znalosti fyziky sú nad rámec stredoškolských osnov,
uvádzame aj stručný sumár potrebnej teórie a~niekoľko predúloh na precvičenie používaných konceptov.
Tá ťažká a~naozaj zaujímavá úloha je predmetom častí ($c$) a~($d$) zadania.
Ako doplnok odporúčame aj 38. kapitolu druhého dielu Feynmanových prednášok z fyziky.}

Makroskopické teleso možno deformovať, pričom sa jeho časť umiestnená v~nedeformovanom stave v~mieste
$\mathbf{r}$ presunie vplyvom deformácie do bodu $\mathbf{r}+\mathbf{u}(\mathbf{r})$.
Deformácia je tak úplne určená vektorovým poľom $\mathbf{u}(\mathbf{r})$. Pre jej určenie
v~danom fyzikálnom probléme je však obvykle výhodné prejsť k~tenzoru deformácie a~tenzoru napätia.

Tenzor deformácie $\varepsilon_{ij}$ je symetrický tenzor \emph{definovaný} ako

\labelmath{
	\varepsilon_{ij}(\mathbf{r}) = \frac{1}{2} \left(\pderive{u_i(\mathbf{r})}{x_j} +
	\pderive{u_j(\mathbf{r})}{x_i} \right)\text{,}
}

kde $x_i$ sú zložky polohového vektora $\mathbf{r}$.
Diagonálne členy $\varepsilon_{xx},\varepsilon_{yy},\varepsilon_{zz}$ zodpovedajú lokálnemu relatívnemu
predĺženiu v~smere osí $x, y, z$ a~mimodiagonálne členy $\varepsilon_{xy}=\varepsilon_{yx}$,
$\varepsilon_{xz}=\varepsilon_{zx}$ a~$\varepsilon_{yz}=\varepsilon_{zy}$ zodpovedajú sklzným deformáciám
v~rovinách $xy$, $xz$ a~$yz$, ktoré pre malé deformácie nevedú k~zmene objemu.
Lokálna kompresia materiálu sa dá pre dostatočne malé deformácie určiť ako 

\labelmath{
	\frac{\Delta V}{V} = \sum_{i=1}^{3} \varepsilon_{ii} =
	\boldsymbol{\nabla}\cdot\mathbf{u} \equiv \kappa\text{.}
}

Deformácia daného telesa je sprevádzaná existenciou vnútorného napätia, ktoré sa prejavuje v~podobe síl
pôsobiacich na povrch myslených objemových elementov. Ak uvážime kúsok vnútra telesa v~tvare kocky
so stranou $\diff a$ s~hranami orientovanými v~smere osí $x,y,z$, tak zložka tenzora napätia
$\sigma_{ij}$ je definovaná ako $j$-ta zložka povrchovej sily pôsobiacej na stenu kocky kolmú na $i$-tu os,
vydelená plochou steny $\mathrm{d} a^2$.

Tieto sily možno chápať ako spôsobené susednými objemovými elementami.
Z~rovnováhy momentov síl pôsobiacich na skúmanú kocku vyplýva symetria tenzora napätia $\sigma_{ij} = \sigma_{ji}$.
Nenulový tenzor napätia môže byť dôsledkom síl pôsobiacich na povrch skúmeného telesa,
alebo dôsledkom kompenzácie objemových síl ako napríklad gravitácie alebo odstredivej sily.

Pre malé deformácie je závislosť medzi tenzorom napätia (podnet) a~tenzorom deformácie (dôsledok) \emph{lineárna}
a~nazýva sa Hookov zákon. Zatiaľ čo pre veľmi anizotropné kryštalické materiály môže byť potrebných
na úplný popis elastických vlastností až 21 elastických konštánt, pre izotropné médium stačia dve.
Častou voľbou sú tzv. Lamého koeficienty $\lambda,\mu$, pomocou ktorých

\labelmath[eqn:2:hooke1]{
	\sigma_{ij}=\lambda \kappa \delta_{ij} + 2\mu\varepsilon_{ij}
}

kde $\kappa$ je lokálna kompresia definovaná vyššie a~$\delta_{ij}=1$ pre $i=j$ a $0$ pre $i\neq j$
je tzv. Kroneckerov symbol. Inverznou rovnicou je 

\labelmath{
	\varepsilon_{ij} = \frac{1}{2\mu} \sigma_{ij} - \frac{\lambda}{2\mu(2\mu+3\lambda)}\delta_{ij}
	\left(\sum_{k=1}^3 \sigma_{kk}\right)\text{.}
}

V~súradniciach, v~ktorých sú oba tenzory diagonálne, možno skrátene písať $\varepsilon_{ii}\equiv \varepsilon_{i}$
a~$\sigma_{ii}\equiv \sigma_{i}$. V~takých súradniciach zodpovedá deformácia zloženej kompresii v~smeroch
$x$, $y$ a~$z$ a~nedochádza k~žiadnej sklznej deformácii ani sklznému napätiu.
Vtedy možno Hookov zákon zjednodušiť na 

\labelmath{
	\sigma_{i}= \lambda \kappa + 2\mu \varepsilon_{i}
}

a~inverznú rovnicu

\labelmath{
	\varepsilon_{i} =  \frac{1}{2\mu} \sigma_{i} - \frac{\lambda}{2\mu(2\mu+3\lambda)}
	\left(\sum_{k=1}^3 \sigma_{k}\right)\text{.}
}

Pre lepšie pochopenie používaných konceptov najprv vyriešte nasledovné dve jednoduché úlohy:
\begin{enumerate}[label=\alph*)]
	\item Uvažujte voľný hranol dĺžky $L$ so základňou tvaru štvorca so stranou $a$, ktorý je naťahovaný silou $F$.
	Vplyvom pôsobiacej sily sa hranol predĺži o~$\Delta L$ a~stenčí o~$\Delta a$. Závislosť medzi
	relatívnym predĺžením $\varepsilon_z=\Delta L / L$, relatívnym zúžením
	$\varepsilon_x = \varepsilon_y = -\Delta a/a $ a~napätím v~ťahu $\sigma_z = F/S$ je lineárna
	\begin{eqnarray}
		\varepsilon_z = \frac{1}{E} \sigma_z = -\frac{1}{\nu} \varepsilon_x = -\frac{1}{\nu} \varepsilon_y\text{,}
	\end{eqnarray}
	kde $E$ sa nazýva Youngov modul a~$\nu$ Poissonov pomer. Vyjadrite elastické konštanty
	$E,\nu$ pomocou Lamého koeficientov $\lambda,\mu$.

	\item Uvažujte hranol rovnakých rozmerov, ktorý je vstavaný do veľmi pevnej steny a~nemôže meniť priečne
		rozmery. Inými slovami, stena pôsobí na bočné steny hranola takými napätiami v~ťahu $\sigma_x=\sigma_y$,
		aby priečna deformácia $\varepsilon_x = 0 = \varepsilon_y$. Nájdite závislosť medzi relatívnym predlžením
		$\varepsilon_z$ a~napätím v~ťahu $\sigma_z$ v~takomto prípade.
\end{enumerate}

Napokon sa dostávame k~tej naozajstnej úlohe:
\begin{enumerate}[label=\alph*)]
	\setcounter{enumi}{2}
	\item Uvažujte asteroid, ktorý má v~nedeformovanom tvare (vo vesmíre bez gravitácie) tvar gule
	s~polomerom $R$ a~hustotou $\rho$. Potom magickým kolieskom zvýšime hodnotu gravitačnej konštanty
	z~nulovej hodnoty na $G$. Určte zmenu polomeru asteroidu $\Delta R$ v~dôsledku gravitačnej interakcie
	so sebou samým. Pri výpočte môžete predpokladať, že $\Delta R$ je dostatočne malé v~porovnaní $R$.

	\emph{Hint: Odvoďte, že kvôli guľovej symetrii problému $\varepsilon_r = \derive{u_r}{r}$
	a~pre zložky rovnobežné s~povrchom $\varepsilon_x = \varepsilon_y = u_r$.
	Pre radiálne posunutie $u_r$ by vám mala vyjsť diferenciálna rovnica
	\labelmath{
%		\der{}{r}\left[\frac{1}{r^2}\der{}{r}\left(r^2 u_r(r)\right)\right] =
%%		\frac{4\pi G \rho_0^2 r}{3(2\mu + \lambda)}\text{.}
		\derive{}{r} \left[ \frac{1}{r^2} \derive{}{r} \left( r^2 u_r(r) \right) \right] =
		\frac{4 \pi G \rho_0^2 r}{3 (2\mu + \lambda)}\text{.}
	}}

	Výsledok pre $\Delta R$ vyjadrite pomocou dvojice Lamého koeficientov $\lambda,\mu$
	ako aj pomocou dvojice elastických konštánt $E,\nu$.

	\item O~koľko sa v~dôsledku tejto deformácie asteroid zohreje, pokiaľ je merná tepelná kapacita
	asteroidu (na jednotku hmotnosti) $c$? Vypočítajte výslednú teplotu pre parametre Zeme a~výsledok
	porovnajte so súčasnou teplotou v Zemskom jadre!

	\emph{Poznámka: Deformácia pre Zem vyjde príliš veľká, teda v~praxi sa už na jej opis lineárna
	elastická teória nevzťahuje. Situáciu tiež komplikuje jej nehomogénne zloženie a~kvapalnosť
	vonkajšieho jadra. Rádová veľkosť výsledku je však napriek týmto problémom zmysluplná a jeho hodnota zaujímavá.}
\end{enumerate}
