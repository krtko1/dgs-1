Robinson a Piatok našli vo fľaši starú nábojovú zbierku. Počas obdobia
dažďov sa im podarilo vyriešiť všetky príklady okrem jedného z
orbitálnej mechaniky, v ktorom zlomyseľní autori chceli aj presný
číselný výsledok. Lenže ak namiesto papiera máte iba paličku a piesok,
deliť gravitačnú konštantu druhou mocninou zemského polomeru nie je
ľahké, takže po šiestich nevydarených pokusoch sa začali navzájom
obviňovať z diletantstva a navyše sa pritom hrozne pohádali, aký ten
polomer Zeme vlastne je.

Po dvoch týždňoch neustáleho škriepenia, keď polovica políčok s jačmeňom
vyschla, papagáje odleteli na tichšiu časť ostrova a nepodojené kozy sa
naučili skákať v ohrade škôlku, Piatka napadlo, že spor predsa môžu
rozsúdiť sami experimentom.

Pomôžte im vymyslieť čo najjednoduchší a zároveň najpresnejší spôsob,
ktorým sa dá odmerať polomer Zeme. Na ostrove majú kanoe, meracie pásmo,
pušky, ďalekohľad a iné bežné veci z roku 1670, prípadne si nejaké
pomôcky môžu v rámci svojich možností vyrobiť. Navyše skúste pouvažovať,
čo by mohli robiť, ak by bolo zamračené, noc, alebo by nemali poruke
more. Nezabudnite dôkladne vysvetliť, prečo váš spôsob funguje a rádovo
odhadnite, akú presnosť je ním možné dosiahnuť.
