\pagestyle{rules}

\section{Pravidlá a postihy}
\begin{itemize}
	\item Seminár je určený pre siedmakov, ôsmakov, deviatakov základných škôl a~sekundánov, terciánov a~kvartánov osemročných gymnázií. 

	\item Cez jeden polrok má seminár 3~série. Všetky série majú po 5~príkladov (každý za 9~bodov), za každú sériu tak môžeš získať 45 bodov.

	\item Siedmaci (sekundáni) a~ôsmaci (terciáni) sú zvýhodnení \emph{bodovou prémiou} (kolonka bonus vo výsledkovej listine)\footnote{%
		Prémia má výšku $\num{0.015} \cdot D \cdot (M - D)$ bodov pre siedmakov a $\num{0.008} \cdot D \cdot (M - D)$ bodov pre ôsmakov,
		kde $D$ je dosiahnutý počet bodov a $M$ je maximálny možný počet bodov v~sérii.}.

	\item Riešenia môžeš posielať poštou alebo elektronicky. Ak posielaš riešenia poštou, pridrž sa nasledujúcich pravidiel:
		\begin{itemize}
			\item každý príklad píš na \emph{osobitný papier A4}, viacstranové riešenie zopni spinkou,
			\item na každý papier napíš hore \emph{hlavičku} s~menom, triedou, školou a~číslom príkladu,
			\item ak riešiš prvýkrát, pošli aj vyplnenú návratku (nájdeš ju na ďalšej strane).
		\end{itemize}
	Viac informácií k~elektronickým riešeniam nájdeš na stránke \URL{\seminarURL} v sekcii e-riešenia.

	\item[$\skull$] Úlohy rieš samostatne! Za odpisovanie strhávame body a~sme agresívni. 

	\item[$\skull$] Príklady posielaj načas! Rozhoduje \emph{termín odoslania} riešení. Riešenia ti zoberieme aj deň po termíne, ale započítaných ti bude len \SI{75}{\percent} získaných bodov.
		Výnimočné prípady riešime individuálne.
\end{itemize}

\subsection{Ako získavať veľa bodov?}
	Ako v~mnohých iných súťažiach, aj tu platí jednoduchá zásada -- písať všetko, čo o~príklade vieš. Teda, aj keď nevieš celé riešenie, oplatí sa písať časti
    riešenia, názory, postrehy, pokusy. Keď chceš vidieť, ako má vyzerať riešenie, prečítaj si \URL{\seminarURL ako\_riesit}. 

	Nemaj strach poslať iba niekoľko úloh. Iba málokto vypočíta všetky úlohy a~dobre umiestniť sa dá aj s~bodmi za menej úloh.

	Píš čitateľne a~tvoje riešenia budú opravené. Píš nečitateľne a~tvoje riešenia budú tiež opravené. Ale predsa by si nás nechcel týrať.

	Ak sa ti nepáči, ako bol príklad obodovaný, pripíš naň rozumný argument, prečo si myslíš že je hodný viac bodov a~pošli späť. Opravovateľ sa zamyslí a~možno aj preboduje.

	Pokiaľ nepochopíš presne zadanie príkladu, môžeš sa e-mailom pýtať na podrobnosti. Pokiaľ máš prístup k~internetu, oplatí sa tiež sledovať debatu zverejnenú na našej stránke (\URL{\seminarURL}).
	Pokiaľ by bola v~príklade nejaká vážnejšia nejasnosť, nebodaj chyba v~zadaní, na debate sa zjaví opravené zadanie príkladu.

	A~hlavne, nenechávaj si príklady na poslednú chvíľu. Skúsenosti potvrdzujú, že za menej ako posledné dve chvíle sa UFO vyriešiť nedá.

\subsection{Riešiť UFO?}
	\begin{itemize}
		\item[+] Spoznáš skvelých ľudí.
		\item[+] Naberieš dačo do hlavy.
		\item[+] Dostaneš sa na sústredko.
		\item[+] Časom môžeš plynule prejsť na stredoškolské kategórie nášho seminára.

		\item[--] Po sústredku ti bude smutno, že bolo také krátke.
		\item[--] Nebudeš môcť spávať od nedočkavosti, kedy ti príde opravená séria.
	\end{itemize}
