\pagestyle{rules}

\section{Pravidlá a postihy}
\begin{itemize}
    \item Seminár je určený pre siedmakov, ôsmakov, deviatakov základných škôl a~sekundánov, terciánov a~kvartánov osemročných gymnázií. 

    \item Počas jedného polroka má seminár \textbf{tri}~kolá. Všetky kolá majú po 5~príkladov (každý za 9~bodov), za každú sériu tak môžeš získať 45 bodov.

    \item Siedmaci (sekundáni) a~ôsmaci (terciáni) sú zvýhodnení \textbf{bodovou prémiou} (kolonka bonus vo výsledkovej
        listine)\footnote{Prémia má výšku $\num{0.015} \cdot D \cdot (M - D)$ bodov pre siedmakov a $\num{0.008} \cdot D \cdot (M - D)$ bodov pre ôsmakov,
        kde $D$ je dosiahnutý počet bodov a $M$ je maximálny možný počet bodov v~kole.}.

    \item Riešenia môžeš posielať iba elektronicky. Ak však stále uprednostňuješ písanie riešenia na papier,
        môžeš svoje riešenie oskenovať a~uložiť každú takto naskenovanú úlohu do jedného \textbf{samostatného} \texttt{pdf} súboru.
        Viac informácií k~riešeniam nájdeš na stránke \URL{http://ufo.fks.sk/pravidla}.

    \item[$\skull$] Úlohy rieš samostatne! Za odpisovanie strhávame body a~sme agresívni. 

    \item[$\skull$] Príklady posielaj načas! Rozhoduje \textbf{termín odoslania} riešení. Riešenia ti zoberieme
        aj deň po termíne, ale započítaných ti bude len \SI{75}{\percent} získaných bodov. Výnimočné prípady riešime individuálne.
\end{itemize}

\subsection{Ako získavať veľa bodov?}
    Ako v~mnohých iných súťažiach, aj tu platí jednoduchá zásada -- písať všetko, čo o~príklade vieš. Teda, aj keď nevieš celé riešenie, oplatí sa písať časti
    riešenia, názory, postrehy, pokusy. Keď chceš vidieť, ako má vyzerať riešenie, prečítaj si \URL{http://ufo.fks.sk/ako\_riesit}. 

    Nemaj strach poslať iba niekoľko úloh. Iba málokto vypočíta všetky úlohy a~dobre umiestniť sa dá aj s~bodmi za menej úloh.

    Ak píšeš riešenia na papier, píš čitateľne a~tvoje riešenia budú opravené. Píš nečitateľne a~tvoje riešenia budú tiež opravené. Ale predsa by si nás nechcel týrať.

    Ak sa ti nepáči, ako bol príklad obodovaný, pripíš naň rozumný argument, prečo si myslíš, že je hodný viac bodov a~pošli späť. Opravovateľ sa zamyslí a~možno aj preboduje.

    Pokiaľ nepochopíš presne zadanie príkladu, môžeš sa e-mailom pýtať na podrobnosti. Pokiaľ máš účet na Facebooku, oplatí sa byť v skupine
    \href{https://www.facebook.com/groups/fks.ufo/}{FKS \& UFO -- Fyzikálne korešpondenčné semináre}.
    Pokiaľ by bola v~príklade nejaká vážnejšia nejasnosť, nebodaj chyba v~zadaní, v novinkách na to upozorníme.

    A~hlavne, nenechávaj si príklady na poslednú chvíľu. Skúsenosti potvrdzujú, že za menej ako posledné dve chvíle sa UFO vyriešiť nedá.

\subsection{Prečo riešiť UFO?}
    \begin{itemize}
        \item[+] Spoznáš skvelých ľudí.
        \item[+] Naberieš dačo do hlavy.
        \item[+] Dostaneš sa na sústredko.
        \item[+] Časom môžeš plynule prejsť na stredoškolské kategórie nášho seminára.

        \item[--] Po sústredku ti bude smutno, že bolo také krátke.
        \item[--] Nebudeš môcť spávať od nedočkavosti, kedy ti prídu opravené riešenia.
    \end{itemize}
