\createTaskHeader[Kondenzácia tuhých gulí]
Predstavme si krabicu, pre jednoduchosť dvojrozmernú. V~nej majme tuhé disky,
ktoré sa pohybujú ako molekuly plynu -- vždy keď na seba narazia, pružne sa od seba odrazia,
inak si voľne cestujú po krabici. 

Intuitívne sa takéto disky, ak ich je len málo, správajú ako plyn (takýto systém sa nazýva \emph{plyn tuhých sfér}). 
Ak však zvýšime hustotu diskov, začnú sa diať veci nevídané -- až tak, že komunita chemikov sa v~50. rokoch
nevedela dohodnúť, čo presne, a~tak sa na jednom kongrese rozhodla ustanoviť pravdu hlasovaním.
To bolo však ešte pred príchodom počítačov. Dnes má každý z~nás k~dispozícii výkonné stroje,
takže si tento problém vieme vyriešiť sami.

\insertPicture{02-hardspheres.png}{6cm}[Plyn tuhých sfér][]

Pri určitej kritickej hustote sa plyn tuhých diskov začne správať kvalitatívne inak -- skondenzuje.
Čo to presne znamená a~pri akej hustote sa to udeje?

Numericky nasimulujte\footnote{Vo svojom obľúbenom jazyku, my odporúčame Python alebo niečo podobne príjemné.}
systém pružne sa zrážajúcich diskov v~krabici (so stenami, alebo ideálne periodickými okrajovými podmienkami)
s~určenou hustotou (definovanou ako podiel plochy zaberanej $N$ diskami o~polomere $R$ v~krabici so stenou dĺžky $L$,
teda  $\rho = N\pi R^2/L^2$). Pomocou vykresľovania grafov polôh diskov možno prísť na to,
ako sa takýto systém správa v závislosti od hustoty. Rozmyslite si,
ako možno fyzikálne vyjadriť (kvantifikovať) \uv{kondenzáciu}.

\subsubsection{Ako na to}
Simulácia sa skladá z nasledujúcich krokov:
\begin{enumerate}[label=\alph*)]
	\item \textbf{Inicializácia}: Vyberte veľkosť krabice $L$, polomer diskov $R$.
		Náhodne umiestnite disky do krabice tak, aby sa neprekrývali (nie vždy to vyjde na prvýkrát).
		Potom každému disku udeľte rýchlosť na základe gaussovského rozdelenia s~hodnotami $\mu=0$ a~$\sigma = T$,
		teda rozumne zvolenou teplotou.
	\item \textbf{Časový vývoj}: Vymyslite, ako navzájom disky pružne zrážať.
		Uistite sa, že to celé fyzikálne dáva zmysel!
	\item \textbf{Vyhodnotenie dát}: Pre každý časový krok zaznamenajte polohu diskov s~časom,
		na základe ktorej možno spočítať priemernú polohu jedného disku a~rozptyl (odchýlku).
		Ako budete postupne zvyšovať hustotu diskov, pri prekročení určite kritickej hodnoty sa
		rozptyl spriemerovaný cez všetky disky výrazne zmení. Čo presne sa stane a pri ktorej hodnote to nastane?
		Ako výsledok simulácie vykreslite závislosť rozptylu od hustoty.
\end{enumerate}

Na Coursere je vynikajúci \href{https://www.coursera.org/course/smac}{kurz Štatistickej mechaniky}
od Wernera Krautha. Odporúčame vám pozrieť si prvé dve-tri časti na zoznámenie sa s~problematikou.
V~prípade nejasností neváhajte napísať na \url{fx@fks.sk}.
