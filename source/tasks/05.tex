\createTaskHeader[Kajkin model trenia]
Kajka sa na poslednom sústredku dozvedela, že trenie je vlastne to, čo
umožňuje ísť okrúhlym veciam vpred. Doma si to chcela otestovať, a tak
vzala valček polomeru $R$ zo stavebnice, nechala ho kotúľať a na oštaru
zistila, že spomaľuje.

Keďže Kajka je motivovaná, rozhodla sa preraziť na poli teoretickej
fyziky a prišla s~vlastným modelom valivého trenia. Predstavila si, že
ten valček je vlastne hranol s~podstavou pravidelného $N$-uholníka a
vypočítala, koľko energie sa stratí pri prevaľovaní cez hrany.

Pre jednoduchosť si povedala, že pri prevalení cez hranu sa po dopade
hranola na podložku všetka energia potrebná na prevalenie premení na
teplo.

Aký by musel byť koeficient šmykového trenia pri šmýkaní kvádra rovnakej
hmotnosti o~podložku, aby spomaľoval rovnako ako valček? Bude tento
model trenia fungovať aj pre nekonečné $N$?

\insertPicture{05.pdf}{30mm}[Kajkin hranol][picture-05]
