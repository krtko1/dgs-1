\createSolutionHeader[Poistka vs. Li-Ion články][Jaro][Jaro]
Zrejme je každému jasné, že na vyriešenie úlohy je potrebné pochopenie
činnosti elektrického zdroja.\footnote{Alebo možno aj nie, ale sme tu preto, aby sme sa niečo naučili. :-)}

Elektrický zdroj je súčiastka, ktorá koná prácu proti elektrickému
poľu, čím udržuje potenciálový rozdiel medzi svorkami, a tým umožňuje
prúdu tiecť obvodom. Bežne sa na výpočty používajú dva abstraktné
modely zdroja, a to ideálny zdroj napätia a ideálny zdroj prúdu. Oba
sa vyznačujú tým, že v~obvode udržujú konštantné napätie, resp. prúd,
bez ohľadu na to, čo na svorky zdroja pripojíme. Ak by sme použili
jeden z~týchto modelov, veľa by sme sa nenapočítali. V~prípade použitia
ideálneho zdroja napätia by sme len do série zapojili dostatočne veľa
zdrojov a obvodom by tiekol ľubovoľne veľký prúd, ľahko vypočítateľný
z~Ohmovho zákona. V~prípade použitia ideálneho zdroja prúdu by sme
to mali ešte jednoduchšie. Obvodom by tiekol taký prúd, na aký je
stavaný zdroj. Lenže reálny zdroj sa správa inak. Už v~samotnom zdroji
dochádza k~istým stratám, ktoré v~tomto prípade nemožno zanedbať,
ako sa to vo väčšine prípadov robí.

Problémom je, že nevieme, čo presne sa v~takom zdroji deje a ako presne
tie straty vyzerajú. Našťastie nás to ale nemusí ani zaujímať. Teória
elektrických obvodov obsahuje dve veľmi užitočné vety, Théveninovu
a Nortonovu teorému\footnote{\url{https://en.wikipedia.org/wiki/Thevenin's\_theorem}, resp.
\url{https://en.wikipedia.org/wiki/Norton's\_theorem}}. 

V~jednoduchosti, Théveninova teoréma nám hovorí, že keď máme ľubovoľne
zložitý lineárny obvod a zaujíma nás prúd tečúci len jednou jeho vetvou,
tak zvyšok obvodu možno nahradiť ideálnym zdrojom napätia s~napätím rovným 
napätiu medzi svorkami nahradenej časti, zapojeným s~odporom do série,
ktorého veľkosť je rovná odporu nahradenej časti.%
\footnote{Nortonova teoréma je jej analógiou, keď ľubovoľnú časť obvodu nahradíme 
ideálnym zdrojom prúdu s~paralelne zapojeným vnútorným odporom. Obe
vety sú si rovnocenné a dá sa medzi nimi jednoducho prechádzať.} 

A~to je presne náš prípad. Reálny zdroj možno považovať za zložitú
časť obvodu, pričom nás zaujíma len prúd tečúci jedinou vetvou (cez
poistku), preto reálny zdroj nahradíme podľa Theveninovej vety ideálnym
zdrojom a k~nemu do série zapojeným vnútorným odporom.\footnote{Použijeme Theveninovu vetu a nie Nortonovu, pretože v~zadaní sa hovorí o~zdroji s~elektromotorickým napätím a nie o~zdroji prúdu.} V~celom príklade bude Li-Ion článok nahradený ideálnym zdrojom napätia
$U=\SI{3.3}{\volt}$ a k~nemu do série pripojeným odporom $R_{i}=\SI{10}{\milli\ohm}$.
Najjednoduchší obvod s~jedným článkom preto bude vyzerať takto:

\insertPicture{task5-z1.pdf}{20mm}[Náhradné zapojenie s~jedným článkom][zapojenie1]

Prúd tečúci týmto obvodom bude jednoducho podľa Ohmovho zákona $I=(U/R_{i}+R)=\SI{165}{\ampere}$.
Vidíme, že jediný článok ani zďaleka nestačí na prepálenie poistky.
Budeme preto musieť zapájať zložitejšie obvody.

Zrejme najistejší spôsob, ako analyzovať zložitejšie obvody, sú Kirchhoffove
zákony. Všetky výpočty budú založené práve na nich, a tak by bolo
na mieste si ich pripomenúť. 

\emph{Prvý Kirchhoffov zákon} hovorí, že súčet
prúdov, ktoré do uzla vtekajú, sa rovná súčtu prúdov, ktoré z~neho
vytekajú. Podľa \emph{druhého Kirchhoffovho zákona} zase algebraický súčet
úbytkov napätí zdrojov v~uzavretej slučke je rovný súčtu napätí na
jednotlivých spotrebičoch danej slučky. 

Priamym dôsledkom Kirchhoffových
zákonov je skutočnosť, že ak nerozvetvenou časťou obvodu tečie prúd
$I$ a v~nejakom mieste sa obvod vetví na $k$ zhodných vetiev, tak
každou z~nich preteká rovnaký prúd $I/k$, čo budeme výhodne
využívať na zníženie počtu rovníc. Viac o~Kirchhoffových zákonoch
tu písať nebudeme, nakoľko všetky návody, ako zostavovať rovnice,
sa dajú ľahko nájsť na webe alebo v~starých príkladoch FKS.

Teraz, keď vieme, ako na to, môžeme sa pustiť do analyzovania jednotlivých
zapojení. Na začiatok preverme dve najbežnejšie zapojenia, sériové
a paralelné.

\insertPicture{task5-z2.pdf}{24mm}[Obvod s~$n$ článkami zapojenými v~sérii][zapojenie2]

Najskôr zapojme do série $n$ článkov. Podľa druhého Kirchhoffovho zákona napíšeme rovnicu
pre zapojenie $nU=\left(nR_{i}+R\right)I$, odkiaľ

$$I=\frac{nU}{nR_{i}+R}=\left.\frac{U}{R_{i}+\frac{R}{n}}\right|_{n\rightarrow\infty}=\frac{U}{R_{i}}\text{.}$$

Tento zápis znamená, že ak $n$ budeme zvyšovať nad všetky medze,
tak prúd sa bude tým viac k~danej hodnote približovať. To preto, lebo
ak $R$ je nejaké reálne číslo a $n$ rastie až do nekonečna, tak
podiel $R/n$ je prakticky rovný nule (presnejšie povedané --
blíži sa k~nule). Pre číselné hodnoty $\underset{n\rightarrow\infty}{\lim}I\left(n\right)=\SI{330}{\ampere}$\footnote{lim je limita a vyjadruje presne to, čo sme popísali; ak $n$ ide
do nekonečna, tak udáva, k~čomu sa blíži $I$}. Vidíme, že maximálny prúd, ktorý možno dosiahnuť sériovým zapojením
článkov, naozaj prepáli poistku. Zistime, kedy sa tak stane. Označme
$I_{max}$ prúd, pri ktorom sa poistka prepáli. Zrejme musí platiť
nerovnosť $I_{max}\leq I\left(n\right)$, kde $I\left(n\right)=nU/(nR_{i}+R)$.
Odtiaľ dostávame, že 

$$n\geq\frac{I_{max}R}{U-I_{max}R_{i}}=10\text{.}$$

Na záver ešte poznamenajme, že pri dostatočne veľkom počte článkov
zapojených do série je prúd v~obvode určený predovšetkým vnútorným
odporom článkov, keďže v~zadaní stojí, že vonkajší odpor obvodu je
rovnaký ako vnútorný odpor jedného článku. Zdroj s~takouto vlastnosťou
sa nazýva mäkký.

\insertPicture{task5-z3.pdf}{5cm}[Obvod s~$n$ článkami zapojenými paralelne][zapojenie3]

Pozrime sa teraz na paralelné zapojenie $n$ článkov. Podľa 2. KZ
s~využitím poznámky o~identických vetvách dostávame rovnicu $U=R_{i}I/n+RI$,
odkiaľ 

$$I=\left.\frac{U}{\frac{R_{i}}{n}+R}\right|_{n\rightarrow\infty}=\frac{U}{R}\text{.}$$

Prúd v~obvode je tentokrát pre dostatočne veľké $n$ určený predovšetkým
vonkajším odporom zapojenia. Takýto zdroj sa nazýva tvrdý. Pre konkrétne
číselné hodnoty zo zadania $\underset{n\rightarrow\infty}{\lim}I\left(n\right)=\SI{330}{\ampere}$.
Aj v~tomto prípade dokážeme prepáliť poistku. Stane sa tak, ak počet
zapojených zdrojov spĺňa nerovnosť $I_{max}\leq I\left(n\right)$,
kde $I\left(n\right)=nU/(R_{i}+nR)$. Odtiaľ dostaneme, že 

$$n\geq\frac{I_{max}R_{i}}{U-I_{max}R}=10\text{.}$$

Sériové i paralelné zapojenie nám dáva rovnaké číselné výsledky.
V~oboch prípadoch sa poistka prepáli pri aspoň 10 článkoch. Rovnaký
výsledok pre oba zapojenia však nie je pravidlo. Vďačíme za to tomu,
že vnútorný odpor článku a odpor záťaže sú rovnaké. Ak by tomu tak
nebolo, tak by sme dostávali rôzne výsledky.

Snáď ešte jedna poznámka. Paralelné zapojenie má tú nevýhodu, že ak
je náhodou niektorý zo zdrojov \uv{slabší}, tak touto vetvou začne
tiecť elektrický prúd proti smeru polarity slabšieho zdroja, čím sa
výrazne znižuje efektivita takéhoto zapojenia, nehovoriac o~tom, že
by mohlo dôjsť k~poškodeniu zdroja.

\insertPicture{task5-z4.pdf}{2cm}[Najjednoduchšie \uv{hybridné} zapojenie pozostávajúce z~troch zdrojov][zapojenie4]

Už sme zistili, kedy sa poistka prepáli pri sériovom a paralelnom
zapojení. Zostáva už len zistiť, či neexistuje nejaké hybridné zapojenie,
ktoré bude fungovať aj pre menší počet zdrojov. Napovedať by nám mohlo
najjednoduchšie takéto zapojenie pozostávajúce z~troch zdrojov (viz obrázok \ref{zapojenie4}). Podľa druhého Kirchhoffovho zákona dostávame rovnicu $2U=\left(R_{i}+R\right)I+R_{i}I/2$,
odkiaľ $I=4U/(3R_{i}+2R)=\SI{264}{\ampere}$. Pre porovnanie, sériovým
a paralelným zapojením troch článkov dostaneme prúd len $\SI{247.8}{\ampere}$.
To je pre nás zlá správa, pretože sme ukázali, že hybridné zapojenie
vážne dokáže vyprodukovať väčší prúd než obyčajné sériové alebo paralelné,
a tak musíme skúšať ďalej.

Preskúmajme teraz všetky zapojenia štyroch článkov. Je ich až 10 a~každé z~nich treba preveriť.%
\footnote{Ak by sme prihliadali na polaritu zapojenia zdrojov, tak ich je ďaleko
viac, no predpokladáme, že netreba nikoho presviedčať o~tom, že zapojenie
zdrojov s~opačnou polaritou bude výrazne neefektívnejšie než s~polaritou
súhlasnou.} Všetky sú zobrazené na nasledujúcom obrázku.

\insertPicture{task5-z5.pdf}{120mm}[Prehľad všetkých zapojení štyroch zdrojov][zapojenie5]

Teraz už len zostáva dopočítať prúd v~jednotlivých obvodoch. Naznačili
sme, ako treba postupovať pri zostavovaní rovníc pomocou Kirchhoffových
zákonov, a~tak tu uvedieme už iba výsledky. Pre väčšiu prehľadnosť
ich zapíšme do tabuľky.

\renewcommand{\arraystretch}{1.6}
\begin{longtable}{@{\extracolsep{\fill}} c c r}
    \toprule
        Číslo obvodu & Algebrický výsledok & Numerický výsledok \\
    \midrule
    \endhead
        1) & $I=\frac{4U}{4R_{i}+R}$ & \SI{264}{\ampere} \\
        2) & $I=\frac{6U}{5R_{i}+2R}$ & $\sim$\SI{283}{\ampere} \\
        3) & $I=\frac{2U}{R_{i}+R}$ & \bfseries{\SI{330}{\ampere}} \\
        4) & $I=\frac{7U}{5R_{i}+3R}$ & \SI{288.75}{\ampere} \\
        5) & $I=\frac{2U}{R_{i}+R}$ & \textbf{\SI{330}{\ampere}} \\
        6) & $I=\frac{6U}{3R_{i}+4R}$ & $\sim$\SI{283}{\ampere} \\
        7) & $I=\frac{7U}{3R_{i}+5R}$ & \SI{288.75}{\ampere} \\
        8) & $I=\frac{6U}{4R_{i}+3R}$ & $\sim$\SI{283}{\ampere} \\
        9) & $I=\frac{6U}{2R_{i}+5R}$ & $\sim$\SI{283}{\ampere} \\
        10) & $I=\frac{4U}{R_{i}+4R}$ & \SI{264}{\ampere} \\
    \bottomrule
    \caption{\textit{Prúdy pretekajúce poistkou pre jednotlivé zapojenia štyroch zdrojov}}
\end{longtable}

Podarilo sa nám nájsť dve také zapojenia (3~a~5) pozostávajúce zo
štyroch článkov, ktoré prepália poistku.\footnote{To, že zapojenia 3~a~5 dávajú rovnaké výsledky, nie je náhoda. Sú
to ekvivalentné zapojenia. Vodič medzi vetveniami obvodu 3 nekladie
žiaden odpor, t.j. oba uzly majú rovnaký potenciál, a teda ich možno
stiahnuť do jedného. Navyše zapojenie je symetrické, teda cez uzol
neprechádza medzi hornou a dolnou vetvou prúd, teda ho možno rozpojiť
a~dostaneme zapojenie 5.} Zároveň sme ukázali, že žiadne zapojenie pozostávajúce z~menšieho
alebo nanajvýš rovnakého počtu článkov nedokáže poistku prepáliť.
Tým je úloha vyriešená.

%asi by bolo fajn doplniť do vzoráku, za čo sa udeľujú body; navrhoval by som 2 body za nejaké zapojenie, ktoré prepáli poistku, +2 za každé optimálne zapojenie a +3 za ukázanie, že sú to všetky optimálne zapojenia
