\createTaskHeader[Obytná štvrť][3][12]
Kráľovstvo Slnka a Paliem sa rozhodlo, že postaví mesto uprostred púšte.
Obytná štvrť bude pozostávať z~$n\times n$ domov usporiadaných do
štvorca. Každá budova bude mať nejaký počet podlaží a všetky podlažia
budú rovnako vysoké.

Na prízemí každého domu bude bývať služobníctvo a na poschodiach vyššia
šľachta\footnote{čím vyššia šľachta, tým vyššie poschodie}. Interiér
budovy je však pre vyššiu šľachtu nepostačujúci, stiesnené priestory
veru nie sú nič pre nich. Aby si mohli užívať čerstvý vzduch, chceli by
mať šľachtici možnosť prejsť sa po streche vedľajšej budovy. Na to však
musí vedľa stáť budova so strechou v správnej výške. Napríklad vedľa
každej štvorpodlažnej budovy musí byť jednopodlažná, dvojpodlažná aj
trojpodlažná budova.

Kráľ by chcel, aby celkový počet podlaží v obytnej štvrti bol čo
najväčší a preto vyhlásil súťaž o najlepší územný plán.

\subsubsection{Úloha}

Vašou úlohou je pre dané $n$ vytvoriť plán mesta. Plán mesta je matica
$n\times n$ čísel predstavujúcich výšky budov. Každé číslo musí susediť
so všetkými menšími kladnými celými číslami, ako je ono samé. Keďže
číslo má najviac štyroch susedov, tak najväčšie možné číslo v~matici je
5.

Počet bodov, ktoré za úlohu dostanete, bude závisieť od celkového súčtu
čísel v~matici.

\subsubsection{Odovzdávanie príkladu}

V tejto úlohe namiesto toho, aby ste odovzdali program, odovzdávate hotové plány mesta.
Zaujímajú nás plány mesta pre nasledovné~$n$: 3, 5, 8, 13, 21, 34, 55, 89 a 144.

Pre každé takéto $n$ vyrobte jeden súbor, ktorý má presne $n$ riadkov a~v~každom
riadku presne $n$ cifier (za ktorými nasleduje znak konca
riadku, čiže \verb|\n|). Cifry vyjadrujú počet podlaží
veže. Tento súbor nazvite \texttt{n.txt}, teda napríklad \texttt{13.txt}
pre $n=13$. Následne tieto súbory všetky zabaľte do jedného zipu a
odovzdajte.

Okrem toho odovzdajte stručný popis toho, ako ste úlohu riešili.

\subsubsection{Hodnotenie}

Každý z 9 odovzdaných plánov sa hodnotí samostatne a za každý môžete
získať $0$ až $\slashfrac{4}{3}$ bodu. Keď váš odovzdaný plán má celkovo o
$d$ podlaží menej ako náš, dostanete ${4}\cdot{3^{-1-d/n}}$ bodov.

Napríklad, ak pre $n=3$ odovzdáte plán s 20 podlažiami, dostanete plný
počet, približne 1.333 bodu. Ak odovzdáte plán s 17 podlažiami,
dostanete 0.444 bodu a~za plán s~19 podlažiami približne 0.924
bodu.

Za popis môžete dostať 0~až~3 body. Nemusíte písať dlhé eseje, stačí
stručne popísať, ako ste úlohu riešili, alebo ako by ste ju chceli
riešiť.

\subsubsection{Príklady}

\exampleIO{
2
}{
11
23
}

\exampleIO{
6
}{
112111
234521
111321
112311
121311
111211
}

\emph{Prvý ukážkový vstup má optimálny počet podlaží, druhý sa dá ešte dosť zlepšiť.}
