\createSolutionHeader[Surfovanie v~MHD][Paťo][Marco]
Zrýchľujúce alebo spomaľujúce metro je z~Maťovho pohľadu tzv. neinerciálna vzťažná sústava.
To je sústava, ktorá sa voči inerciálnej sústave (napríklad voči nástupišťu) pohybuje s~nejakým zrýchlením.

V~neinerciálnych sústavach platí, že okrem síl, ktorých pôvod dôverne poznáme,
bude na Maťa pôsobiť ešte ďalšia, tzv. fiktívna sila. V~našom prípade sa sila, ktorá
na Maťa pôsobí v~zrýchľujúcom metre, nazýva \emph{zotrvačná} sila. Jej veľkosť
je $F_{\mathrm{z}} = ma$, kde $a$ je veľkosť zrýchlenia metra. Smer zotrvačnej sily je
presne opačný ako smer zrýchlenia vzťažnej sústavy.

Aby sme mohli vypočítať maximálne možné $a$, musíme si -- najlepšie do obrázku -- zakresliť
všetky sily, ktoré na Maťa pôsobia. Predpokladajme, že metro zrýchľuje smerom
doprava, takže zotrvačná sila musí pôsobiť doľava. Aby Maťo rovno nespadol,
váhu (ťažisko) preniesol na svoju pravú nohu (viď obrázok \ref{sily}).

Ďalej na Maťa pôsobí tiažová sila (v~ťažisku, smerom kolmo
nadol). Na chodidlá naňho pôsobí podlaha metra silami $F_1$ a $F_2$ (ich veľkosť
nepoznáme, vieme iba, že pôsobia kolmo na podlahu metra, tzn. kolmo nahor).

Úplne nakoniec na Maťa pôsobia trecie sily $T_1$ a $T_2$. Keďže sila $F_{\mathrm{z}}$ chce
Maťa \uv{ťahať} doľava, trecie sily sa budú snažiť tomuto pohybu brániť a pôsobiť
smerom doprava (viď obrázok \ref{sily}). Veľkosť trecích síl je
z~definície $T_1 = f F_1$ a $T_2 = f F_2$, kde $f$ je trecí koeficient.

\Picture{task3.pdf}{3cm}[Sily pôsobiace na Maťa][sily]

Situácia na obrázku vyzerá optimisticky -- všetky zakreslené sily pôsobia
buď vo vodorovnom alebo v~zvislom smere. Okamžite preto môžeme napísať
rovnice pre rovnováhy síl v~týchto smeroch:
$$F_1 + F_2 = mg$$
$$T_1 + T_2 = F_{\mathrm{z}}\text{.}$$

Ak do rovníc dosadíme $F_{\mathrm{z}} = ma$ a vyjadrenia pre trecie sily, z~rovníc
rýchlo dostávame podmienku
$$
  f mg = ma \qquad\Rightarrow\qquad a = fg \text{.}
$$
To zamená, že ak bude zrýchlenie metra väčšie ako $ 0.5g \approx \SI{5}{\meter\per\second\squared}$, Maťo,
nech sa akokoľvek snaží nespadnúť, sa začne na podlahe metra šmýkať v~protismere
jeho jazdy.

Vyzerá to tak, že sme úlohu vyriešili. Bohužiaľ, stále Maťovi nevieme zaručiť
bezpečnú cestu. Musíme sa ešte pozrieť na to, či náhodou Maťo pôsobením
zotrvačnej sily nespadne skôr, ako príde k~scenáru popísanému vyššie.

Pozrime sa preto ešte na momenty síl, ktoré sú úzko spojené s~otáčaním telies.
Aby Maťo nepadal ($=$ neotáčal sa), musia byť v~rovnováhe aj momenty síl,
a to okolo \emph{ľubovoľnej} osi. To nám poskytuje obrovskú výhodu -- os otáčania
si môžeme zvoliť tak, aby rovnica rovnováhy momentov síl bola čo najjednoduchšia, tzn. obsahovala
čo najmenej neznámych síl. V~našom prípade sú neznáme sily $F_1$ a $F_2$ (a sily
$T_1$ a $T_2$). \uv{Zbavíme sa} ich tak, že os otáčania zvolíme do takého bodu, že
tieto sily, resp. ich väčščina bude mať nulový moment.

Ideálni kandidáti na tieto body sú Maťove chodidlá. Pozrite sa na obrázok
\ref{sily}. Ak zvolíme os otáčania v~Maťovej pravej nohe, len dve sily
majú nenulový moment (sily $F_2$ a $F_{\mathrm{z}}$). Platí teda
$$
  F_2 x = F_{\mathrm{z}} y\text{.}
$$
Ak si predstavíme os otáčania v~ľavej Maťovej nohe, rovnakou úvahou dospejeme
k~rovnováhe momentov
$$
  F_1 x + mg x = F_{\mathrm{z}} y\text{.}
$$
Ak rovnice sčítame, dostaneme (po úprave a dosadení z~rovnováhy síl $F_1 + F_2 = mg$)
rovnicu
$$
   a = \frac{y}{x} g\text{.}
$$
Po dosadení za $y = (3/5) \cdot \SI{185}{\cm} = \SI{111}{\cm}$ a $x = \SI{40}{\cm}$
dostaneme $a \approx \SI{0.36}{\gforce} = \SI{3.6}{\meter\per\second\squared}$, čo je menej, ako predošlý výsledok.

Metro teda nemôže zrýchľovať viac, ako zrýchlením $\SI{3.6}{\meter\per\second\squared}$.
Po prekročení tejto hranice Maťo pôsobením príliš veľkého momentu zotrvačnej
sily spadne skôr, ako sa jeho topánky začnú šmýkať po podlahe.

Všimnite si, že v~celom riešení sme nepotrebovali vedieť, aké veľké sú sily
$F_1$ a $F_2$. Tento poznatok sme obišli dôsledným zápisom rovnováhy síl
a~prefíkanou voľbou osí otáčania. 
