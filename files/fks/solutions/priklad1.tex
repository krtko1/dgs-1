\createSolutionHeader[Vážna úvaha][Dušan][Dušan]
Tak som sa teda zamyslel, a predsa len som na niečo prišiel.

Najprv si povieme niečo o~váhe a~o~tom, ako funguje.
To, čo naozaj meriame váhou, je sila, ktorou na ňu pôsobíme, nie hmotnosť.
My však poznáme prevod medzi našou hmotnosťou a tiažovou silou.
Konkrétne $F = mg$, kde $g$ je tiažové zrýchlenie, čiže žiaden problém.
Otázkou teda zostáva: ,,ako zmeriame silu?``
Štandardne sa používa pružina, lebo sa stláča rovnomerne\footnote{teda vo väčšine prípadov}.
To znamená, že výchylka pružiny od rovnovážnej polohy $x$ je priamo úmerná pôsobiacej sile, čo zapíšeme ako
$F = k x$, kde $k$ je tuhosť pružiny. Keď si dáme dve a dve dohromady, prídeme na to, že 
v~takej pružinovej váhe stačí vyrobiť ciferník, ktorý ukáže hmotnosť $(k/g)x$,
ak sa pružina váhy stlačí o~$x$ po tom, čo sa na ňu postavíme.

Len tak pre zaujímavosť spomeniem, že dnes sa štandardne stretnete s~elektronickými váhami.
Tie fungujú na podobnom princípe ako pružinové, no miesto pružiny sa tam používa piezoelektrický kryštál.
To je taký materiál (konkrétne dielektrikum), v~ktorom sa pri mechanickej deformácii vytvára elektrický náboj.
Ten vieme jednoducho merať a ak poznáme prevod medzi týmito veličinami, pomerne ľahko vieme postaviť funkčnú váhu.

Teraz k~tej druhej, zaujímavejšej otázke: ,,Ako zmerať hmotnosť v~beztiažovom stave wtf ``,
teda vtedy, keď tiažovú silu merať nemôžeme.
Možností je viacero, no najkrajší bude asi príklad z~praxe. V~ňom sa využíva tiež pružina, ale merajú sa jej kmity.
Konkrétne sa na ňu zavesí objekt, ktorého hmotnosť chceme merať, uvedie sa do pohybu a meria sa 
perióda jeho kmitov. Ak by sme chceli zistiť, aký je prevod medzi hmotnosťou a periódou,
museli by sme vyriešiť pohybovú rovnicu
$$m a(t) = -k x(t)\text{.}$$ 
Jej riešenie je však pomerne známe, lebo ide jedná o~lineárny harmonický oscilátor,
ktorého perióda kmitov je
$$T = 2 \pi \sqrt{\frac{m}{k}}$$
a teda hmotnosť vieme vyjadriť ako
$$m = \dfrac{kT^2}{4 \pi^2}\text{.}$$

No a~toto je presne to, čo sme hľadali. Samozrejme, medzi vašimi riešeniami sa našli aj iné spôsoby, nie len tento.
Nebojte sa, ak boli fyzikálne správne, boli odmenené plným počtom bodov.
