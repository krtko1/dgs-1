%Niekto by si to mohol pozorne prečítať a zhodnotiť potrebnosť/pridanú hodnotu insightov vo footnotoch, či ich netreba viac alebo naopak nie sú zbytočné ...
Najprv si pripomeňme, čo je to vlastne \emph{tepelná kapacita}. Tá nám hovorí, koľko energie vo forme tepla musíme
\lq\lq{zaplatiť}\rq\rq{} za to, aby sme ohriali systém o~jednotku teploty (napríklad jeden stupeň Celzia).\footnote{Zdôrazňujeme však, že formálne nie je $\Delta$ pri $Q$ to isté ako $\Delta$ pri $T$, keďže teplo nie je \textbf{stavová veličina}!
A~preto by sme ich mali po správnosti rozlišovať.}
$$
  C \equiv \frac{\Delta Q}{\Delta T} \text{.}
$$
% Tu som si nie istý, či nechceme zdôrazniť, že nie je Delta ako Delta, no keďže je to prvá séria, tak to asi nechceme siliť ...
% Predsa len som sa teda rozhodol čosi napísať.
%Takto vo footnote to stačí. Teda ja pochybujem, že väčšina deciek si neuvedomuje, že čo sú stavové veličiny...Dušan

Teplo sa môže uložiť do vnútornej energie plynu alebo použiť na vykonanie práce. V~našom prípade môže plyn tlačiť na piest,
čím vykoná stlačením pružinky piestu prácu. Dôležité ale je, aby sme si uvedomili, že práca, ktorú vykoná plyn tlačením na
piest, je rovnaká ako nárast potenciálnej energie pružinky. Preto tento príspevok energie môžeme započítať iba raz!\footnote{Keďže táto energia je uložená v~systéme \lq\lq{iba raz}\rq\rq{} v~podobe potenciálnej energie pružinky.}
Teplo by sa mohlo ešte teoreticky použiť na ohriatie materiálu stien nádoby, piestu a pružinky, no v~zadaní sme povedali,
že ich tepelné kapacity sú zanedbateľné. Každopádne, energia sa musí zachovať, čo môžeme zhrnúť \emph{zákonom zachovania
energie}, v~termodynamike taktiež nazývaným aj ako \emph{prvá veta termodynamická}, ktorý môžeme následne dosadiť do definície
tepelnej kapacity:
  $$ \Delta Q = \Delta U_{\text{plyn}} + \Delta W $$
  $$C = \frac{\Delta U_{\text{plyn}}}{\Delta T} + \frac{\Delta W}{\Delta T} \text{.}$$

Zo zadania ďalej vieme, že piest (pružinka) sa nachádza v~pokojovom stave, keď sa dotýka pravej steny nádoby. Ak si označíme 
plochu piestu $S$, objem plynu v~pravej časti piestu $V$ a dĺžku pravej časti s~plynom $x$, čo je zároveň stlačenie pružinky, tak 
vieme, že sila, ktorou pôsobí plyn na piest $F_{\text{plyn}}=pS$, musí byť rovnaká ako sila, ktorou pôsobí pružinka na piest 
$F_{\text{pruž}} = kx$. Odtiaľ
$$F_{\text{pruž}} = F_{\text{plyn}}$$
$$kx = pS$$
$$kx^2 = pSx = pV\text{.}$$

Keďže vieme, že plyn je jednoatómový a môžeme ho považovať za ideálny, môžeme rovno pre vnútornú energiu plynu 
napísať známy vzťah $\Delta U = (3/2)nR\Delta T$.\footnote{Konštnata tri súvisí s~\emph{počtom stupňov voľnosti} molekúl plynu.
Jednoatómový plyn má iba tri, keďže každý atóm sa môže pohybovať v~troch smeroch a pri ľubovoľnej rotácii okolo svojho stredu
je nerozoznateľný. Na stupeň voľnosti sa môžete pozerať ako na počet parametrov, ktoré potrebujeme poznať na to, aby sme
presne určili pozíciu a natočenie molekuly plynu. Skúste si teda premyslieť, koľko stupňov voľnosti majú dvojatómové molekuly.}

Zostáva nám teda zistiť, akú nekonečne malú prácu $\Delta W$ vykoná plyn, keď sa zväčší teplota plynu o~nekonečne malý 
prírastok $\Delta T$. Práca $\Delta W$ je rovná $p\Delta V = pS\Delta x = kx\Delta x$. Vložením zatiaľ všetkého, čo sme zistili, do 
\emph{prvej vety termodynamickej} dostaneme:
$$C = \frac{\Delta U_{\text{plyn}}}{\Delta T} + \frac{\Delta W}{\Delta T} \hspace{3.5cm}
C = \frac{3}{2}nR + kx\frac{\Delta x}{\Delta T} \,\text{.}$$

Mali by sme taktiež vedieť, ako súvisí $\Delta x$ s~$\Delta T$. Na to si pomôžeme stavovou rovnicou ideálneho plynu. Označme $n$ počet 
mólov plynu. Ďalej vieme, že v~každom okamihu musí platiť rovnica ideálneho plynu, t.j. aj po ohriatí o~$\Delta T$.\footnote{Na 
systém sa pozeráme po ohriatí o~$\Delta T$ vždy, keď sa ustáli v~stave termodynamickej rovnováhy, pre ktorý stavová rovnica
v~prípade ideálneho plynu platí.}
$$kx^2  = pV = nRT $$
$$k{\left(x+\Delta x\right)}^2 = nR\left(T+\Delta T\right)\text{.}$$

Odčítaním týchto dvoch rovníc a zanedbaním členu $k{\Delta x}^2$\footnote{Keďže vo fyzike nás často zaujíma ako sa veci majú 
do \emph{prvého rádu}, tak si to môžeme dovoliť. Intuitívne si to môžete predstaviť, že ak je $\Delta x$ malé, tak ${\Delta x}^2$ 
je o~niekoľko rádov menšie
 a teda oproti $\Delta x$ zanedbateľné.}, šikovne zistíme ako súvisí $\Delta x$ s~$\Delta T$.
$$k{\left(x+\Delta x\right)}^2 - kx^2 \approx 2kx\Delta x = nR\Delta T \hspace{3cm}
\frac{\Delta x}{\Delta T} = \frac{nR}{2kx}\text{.}$$

Pokročilejší z~vás sa mohli vyhnúť tejto obskúrnosti vyjadrením $\Delta T$ a $\Delta x$, diferencovaním výrazu $kx^2 = nRT$,
čím by sme ale dostali rovnaký výsledok.

Dosadením všetkého, čo sme sa dozvedeli, do definície tepelnej kapacity dostaneme celkom milý výsledok:
$$
C = \frac{3}{2}nR + \frac{1}{2}nR = 2nR = 2\frac{p_0 V_0}{T_0} \text{,}
$$
kde sme nakoniec vyjadrili výraz $nR$ na základe veličín zo zadania, keďže počet častíc plynu sa v~pieste nemení.