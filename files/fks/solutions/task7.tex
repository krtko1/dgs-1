\createSolutionHeader[Nezelené hviezdy][Maťo B][Maťo B]
Predtým, než si zodpovieme otázku zo zadania, si vyjasnime, čo je to vlastne \emph{farba}. 

\emph{Farba} je čisto subjektívny pocit a interpretácia svetla istej vlnovej dĺžky našim okom a mozgom.
Naše oči sa skladajú z~troch druhov čapíkov, tzv. S, M a~L čapíky, ktoré sú rozdielne citlivé na svetlá rôznych vlnových dĺžok.
Pozor, tieto čapíky nie sú citlivé na svetlo iba jednej vlnovej dĺžky, ale reagujú na pomerne široký rozsah vlnových dĺžok.
Pri veľmi letmom pohľade tieto čapíky korelujú s~voľbou troch základných farieb, ktoré sa používajú na monitoroch a displejoch -- RGB\footnote{Red, green, blue, teda červená, zelená a modrá.}.
To, akú farbu vidíme, teda záleží od dvoch vecí: citlivosti nášho oka na svetlo rôznych vlnových dĺžok a samotného spektra svetla,
ktoré dopadá do nášho oka, čiže toho, ako veľmi sú zastúpené rôzne vlnové dĺžky v~dopadajúcom svetle.

Spektrum svetla, resp. žiarenia hviezd možno pre naše účely považovať za spektrum \emph{absolútne čierneho telesa}.
Hviezdy však žiaria iba v~obmedzenom spektre, ktoré súvisí s~ich zložením a zodpovedá preskokom 
elektrónov medzi rôznymi energetickými hladinami v~atómoch a molekulách, ktoré hviezdy tvoria.
Ďalej budeme tento fakt veselo ignorovať a budeme hviezdy považovať za absolútne čierne telesá. Áno, viem ako zvláštne to znie :). No na to ste si už asi vo fyzike zvykli. Pravdou je, že pre väčšinu hviezd je takáto aproximácia dostatočná.
Aj keby sme prítomnosť spektrálnych čiar uvažovali, na výsledok by to nemalo veľký vplyv.
Hviezda by totiž musela mať naozaj \emph{veľmi exotické} zloženie, aby to bolo dôvod na to, že ju vidíme zelenú. 

\Picture{fraunhofer.pdf}{3cm}[Obrázok \emph{Fraunhoferových čiar} -- spektrálnych čiar v~spektre Slnka.%
%\footnote{Obrázok je prepožičaný z~\url{https://upload.wikimedia.org/wikipedia/commons/2/2f/Fraunhofer_lines.svg}.}
][fraunhofer]

Žiarenie absolútne čierneho telesa sa riadi tzv. \emph{Planckovým vyžarovacím zákonom}\footnote{\url{https://en.wikipedia.org/wiki/Planck's_law}}.
Ten nám hovorí, koľko elektromagnetického žiarenia vyžaruje absolútne čierne teleso pri určitej teplote. 
$$
I(\lambda) = \frac{2hc^2}{\lambda^5}\frac{1}{e^{\frac{hc}{\lambda k T}}-1}\text{,}
$$
kde $I$ označuje vyžiarený výkon na jednotku plochy, priestorový uhol a vlnovú dĺžku $\lambda$, $h$ je Planckova konštanta, $k$ je Boltzmannova konštanta, $c$ je rýchlosť svetla a $T$ je teplota absolútne čierneho telesa.

\emph{Wienov posuvný zákon},
$$
  \lambda_{\text{max}} = \frac{\SI{2.898}{\milli\metre\kelvin}}{T}\text{,}
$$
nám pre každú teplotu hovorí, na ktorej vlnovej dĺžke $\lambda$ vyžaruje absolútne čierne teleso najviac žiarenia (teda pre aké $\lambda$ je $I(\lambda)$ maximálne). Vieme z~neho ľahko dopočítať, akú teplotu by musela mať hviezda, aby najviac žiarenia vyžarovala na vlnovej dĺžke $\lambda = \SI{530}{\nano\metre}$, čo je približne zelená.
Ľahkým dosadením zistíme, že hviezda by musela mať teplotu $T \approx \SI{5470}{\kelvin}$, čo vôbec nie je nezvyčajná teplota pre hviezdu! 

Ako to, že teda nepozorujeme na oblohe zelené hviezdy? Hviezdy, ktoré vyžarujú najviac žiarenia v~zelenej časti spektra, vyžarujú veľa žiarenia aj vo zvyšku viditeľnej časti spektra.
Keďže viditeľné spektrum je naozaj pomerne úzke oproti celému vyžarovanému spektru hviezdy, \uv{zelené} (M) čapíky sú dráždené podobne, ako zvyšné dva druhy čapíkov.
Farbu určuje to, ako veľmi podobne to je. Dokonca ak je teplota hviezdy vyššia ako $\SI{20000}{\kelvin}$, svetlo hviezdy vnímame ako modrasté bez ohľadu na to, aká je teplota hviezdy,
keďže najviac sú dráždene práve \uv{modré} čapíky.

Toto je dôvod, prečo astronómovia nepozorujú zelené hviezdy.\footnote{V skutočnosti sa im občas stane,
že uvidia zelenú hviezdu, ale to iba v~skupine červených hviezd. To je však spôsobené len únavou nášho oka, ktoré v~prítomnosti veľa červených bodiek začne vidieť aj zelené.}

Už vieme, prečo nie sú hviezdy zelené, no musíme ešte zistiť, akú farbu majú vlastne hviezdy, ktoré by \uv{mali byť zelené}.
Môžeme použiť buď tento obrázok\footnote{\url{https://upload.wikimedia.org/wikipedia/commons/b/ba/PlanckianLocus.png}}

\Picture{planck.png}{8cm}[Na obrázku môžeme vidieť krivku, na ktorej je znázornená teplota v~Kelvinoch. A~príslušné farby, ktoré zodpovedajú príslušnej teplote absolútne čierneho telesa.][planck]

alebo si to skúsime sami vypočítať. Na to, aby sme to vypočítali, však musíme vedieť, ako presne sú naše čapíky citlivé na svetlo rôznych vlnových dĺžok.
Našťastie nemusíme dlho hladať, pretože ak sa trošku posnažíme, dopátrame sa k~tzv.
\emph{CIE color matching functions}\footnote{\url{https://en.wikipedia.org/wiki/CIE_1931_color_space\#Color_matching_functions}},
čo sú štandardizované citlivosti čapíkov na rôzne vlnové dlžky. Samotné tabuľky $x$, $y$ a $z$ funkcií možno nájsť napríklad tu\footnote{\url{http://cvrl.ioo.ucl.ac.uk/cmfs.htm}}.

Stačí si teda otvoriť tabuľkový procesor, nakopírovať si tabuľky $x$, $y$ a $z$ funkcií (napr. po $\SI{0.1}{\nano\metre}$), prenásobiť vo viditeľnom spektre spektrom
absolútne čierneho telesa teploty $T \approx \SI{5470}{\kelvin}$ a následne sčítať po malých kúskoch (napr. po $\SI{0.1}{\nano\metre}$)
(viď článok o~\emph{Color matching functions}).\footnote{Ak vám táto myšlienka sčitovania po malých kúskoch nie je jasná,
odporúčame si prečítať v~Archive FKS vzorák k~úlohe Vlnitá panoráma z~29. ročníka.} Nakoniec už len nájsť spôsob ako čísla \emph{color functions} $X$, $Y$ a $Z$ prepočítať na RGB farbu. 

Iný spôsob bol priamo nájsť približné vzťahy\footnote{\url{https://en.wikipedia.org/wiki/Planckian_locus\#Approximation}} na výpočet $X$ a $Y$ súradníc v~$CEI$ farebnom spektre priamo z~teploty a potom prepočítať na RGB farbu.

Či už zvolíme prvý alebo druhý spôsob, dopátrame sa k~skoro čisto bielej farbe\footnote{[0.333, 0.342] v~CEI 1931 Color space}.
\uv{Zelené hviezdy} sú teda v~skutočnosti biele. Pravdou je, že v~tom obrovskom množstve hviezd sa môže nachádzať zopár hviezd s~takým exotickým zložením, že vďaka absorbčnému a emisnému spektru prvkov, z~ktorých sa bude skladať, sa nám bude zdať zelená, no žiadnu sme doteraz nenašli. Ak len predsa len narazíte na nejaký obrázok či fotku v~hviezdy v~zelenej farbe, napr. na stránke NASA, ide o~počítačovo upravenú fotku s~umelými farbami.
