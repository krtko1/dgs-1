\createTaskHeader[Elektrický burger]
Maťo sa minule hral s~plošným kondenzátorom s~dĺžkou $l$ a šírkou $s$, ktorého dosky boli od seba vzdialené $h$. 
Medzi doskami kondenzátora bol pôvodne vzduch. Maťo kondenzátor najskôr nabil, takže medzi doskami kondenzátora bolo napätie $U$ a~potom kondenzátor odpojil od zdroja napájania. Maťo sa následne pokúšal do kondenzátora vložiť dielektrickú platničku 
s~relatívnou permitivitou $\varepsilon_r$ a~hrúbkou $h$.

Prečo na to musel vynaložiť silu? Akú veľkú? Ako by sa zmenil výsledok, ak by doštička bola vo vnútri kondenzátora 
a~Maťo by sa ju pokúšal vybrať? Zmenil by sa výsledok, ak by bol kondenzátor stále pripojený na zdroj napájania s~konštantným napätím $U$?

\insertPicture{7.pdf}{6cm}[Platnička vsúvaná medzi dosky kondenzátora][platnicka]
