\createTaskHeader[Spaľovanie informácie]
Mišo začal uvažovať, že si doma postaví stroj, ktorý by spaľovaním informácie vedel konať prácu.
Mišo má informáciu uloženú na pomerne neštandardnom médiu, páske pozostávajúcej z~krabíc rovnakej veľkosti,
pričom v~každej sa nachádza jeden atóm ideálneho plynu.

\insertPicture{paska.png}{15mm}[Mišovo médium][fig:3]

Každá krabica je rozdelená priehradkou na dve rovnaké časti. Priehradku vieme kedykoľvek vybrať.
Horná a~dolná stena krabice sú piesty, ktoré sa pohybujú bez trenia a~vedia tak konaním práce tlačiť
atóm do požadovanej časti krabice. Ak sa atóm nachádza v hornej časti krabice rozdelenej priehradkou,
tak máme uloženú jednotku, ak sa atóm nachádza v~dolnej časti krabice, tak máme uloženú nulu. 
 
Všimnite si, že ak vieme, v~ktorej časti krabice sa atóm nachádza, tak vieme posunúť piest v~prázdnej
časti krabice smerom k~priehradke, odstrániť priehradku a~nechať piest vrátiť sa do pôvodnej pozície.
Zničíme tak informáciu, kde sa nachádza atóm, no vieme tak získať užitočnú prácu.

\insertPicture{spalenie.png}{15mm}[Mišovo médium][fig:3:2]

\begin{enumerate}[(a)]
	\item Predpokladajme, že plyn sa rozpína pri konštantnej teplote $T$.
		Aké množstvo práce vieme získať \uv{spálením} jedného bitu informácie?
	\item Ukážte, že zmenu stavu krabice (z~nuly na jednotku alebo opačne) vieme urobiť vratne a~zadarmo,
		ak vieme, v~akom stave sa nachádza atóm v krabici.
	\item Preštudujte si koncept entropie. Vypočítajte, ako sa zmení entropia systému,
		keď \uv{zabudneme informáciu}, t. j. keď odstránime priehradku. Pri výpočte použite koncept
		termodynamickej entropie, a~potom aj informačnej entropie.
	\item Predstavme si škriatka, \uv{Maxwellovho démona}, ktorý by vedel konať prácu aj takým spôsobom,
		že by nemusel vedieť, v~akom stave sa nachádza krabica (teda kde je atóm v krabici).
		Čo zabraňuje škriatkovi konať prácu takýmto spôsobom? Mišo navrhuje, že na odmeranie stavu,
		v~ktorom sa nachádza atóm, musíme vykonať istú minimálnu prácu. V~skutočnosti sa to však dá aj úplne zadarmo!
		Navrhnite, ako by to išlo urobiť zadarmo a~vysvetlite, prečo sa pri tomto spôsobe neporušuje druhý
		termodynamický zákon. A~prečo teda vlastne škriatok nevie konať prácu takýmto spôsobom?
\end{enumerate}
